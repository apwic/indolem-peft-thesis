%--------------------------------------------------------------------%
%
% Berkas utama templat LaTeX.
%
% author Petra Barus, Peb Ruswono Aryan, Faris Rizki Ekananda
%
%--------------------------------------------------------------------%
%
% Berkas ini berisi struktur utama dokumen LaTeX yang akan dibuat.
%
%--------------------------------------------------------------------%

\documentclass[bahasa, 12pt, a4paper, onecolumn, oneside, final]{report}

%-------------------------------------------------------------------%
%
% Konfigurasi dokumen LaTeX untuk laporan tesis IF ITB
%
% @author Petra Novandi
%
%-------------------------------------------------------------------%
%
% Berkas asli berasal dari Steven Lolong
%
%-------------------------------------------------------------------%

% Ukuran kertas
\special{papersize=210mm,297mm}

% Setting margin
\usepackage[top=2cm,bottom=2cm,left=4cm,right=3cm]{geometry}
% Underline
\usepackage[normalem]{ulem}

\usepackage{mathptmx}

% Judul bahasa Indonesia
\usepackage[bahasa]{babel}

% Format citation
\usepackage[utf8]{inputenc}
\usepackage[style=apa,backend=biber]{biblatex}

\usepackage{graphicx}
\usepackage{titling}
\usepackage{blindtext}
\usepackage{sectsty}
\usepackage{chngcntr}
\usepackage{etoolbox}
\usepackage{array}
\usepackage[hidelinks]{hyperref}       % Package untuk link di daftar isi. Ubah jadi \usepackage[hidelinks]{hyperref} apabila ingin menghilangkan kotak merah disekitar link
\usepackage{titlesec}       % Package Format judul
\usepackage{titletoc}       % Package Format judul di toc
\usepackage{tocbibind}      % Package untuk masukkan toc, lot, lof ke Daftar Isi
\usepackage{scrwfile}       % Package untuk membuat Daftar Lampiran dari toc
\usepackage{parskip}
\usepackage{afterpage}
\usepackage{relsize}
\usepackage{longtable}
\usepackage{xcolor, colortbl}
\usepackage{setspace}
\usepackage{booktabs}
\usepackage{pgfgantt} 
\usepackage{geometry}	% for page margin settings
\usepackage{pdflscape}	% for landscape orientation
\usepackage{enumitem}	% itemize
\usepackage{multirow}	% joining row
\usepackage{xspace}	% spacing for macros
\usepackage{caption}	% table and figure caption spacing
% Daftar Istilah
\usepackage{tabularx}
% Codeblocks
\usepackage{listings}
% Dir tree
\usepackage{tikz}
\usepackage{forest}
% Equation
\usepackage{amsmath}

\graphicspath{{resources/}}   % letak direktori penyimpanan gambar

% Setting daftar lampiran
\newcommand*{\lopname}{DAFTAR LAMPIRAN}
\TOCclone[\lopname]{toc}{atoc}
\addtocontents{atoc}{\protect\value{tocdepth}=-1}
\newcommand\listofappendices{
  \cleardoublepage
  \phantomsection
  \listofatoc
  \addcontentsline{toc}{chapter}{\lopname}
}

\newcommand*\savedtocdepth{}
\AtBeginDocument{%
  \edef\savedtocdepth{\the\value{tocdepth}}%
}

\let\originalappendix\appendix
\renewcommand\appendix{%
  \originalappendix
  \cleardoublepage
  \addtocontents{toc}{\protect\value{tocdepth}=-1}%
  \addtocontents{atoc}{\protect\value{tocdepth}=\savedtocdepth}%

  \titlecontents{chapter}
    [0pt]
    {\bfseries}
    {Lampiran \thecontentslabel.\quad}
    {}
    {\hfill\contentspage}

  \titleformat{\chapter}[block]
    {\bfseries}
    {\chaptertitlename\ \thechapter.\quad}{0pt}
    {\bfseries}
}

% Hilangkan titik pada toc
\makeatletter
\renewcommand{\@dotsep}{1} 
\makeatother

% Setel title pada chapter-chapter di toc, lof, lot
\titlecontents{chapter}
  [0pt]
  {\bfseries}
  {\MakeUppercase{Bab} \thecontentslabel\quad\uppercase}
  {}
  {\mdseries\titlerule*[0.35em]{.}\bfseries\contentspage}
\titlecontents{figure}
  [0pt]
  {}
  {Gambar \thecontentslabel.\quad}
  {}
  {\mdseries\titlerule*[0.35em]{.}\bfseries\contentspage}
\titlecontents{table}
  [0pt]
  {}
  {Tabel \thecontentslabel.\quad}
  {}
  {\mdseries\titlerule*[0.35em]{.}\bfseries\contentspage}

% Masukin Daftar Pustaka ke toc
\let\originalprintbibliography\printbibliography
\renewcommand\printbibliography{%
  \phantomsection
  \cleardoublepage
  \originalprintbibliography
  \addcontentsline{toc}{chapter}{\bibname}
}

% Line satu setengah spasi
\renewcommand{\baselinestretch}{1.5}

% Setting judul
\chapterfont{\centering \large}
\titleformat{\chapter}[display]
  {\Large\centering\bfseries}
  {\chaptertitlename\ \thechapter}{0pt}
    {\Large\bfseries\uppercase}

% Setting nomor pada subbsubsubbab
\setcounter{secnumdepth}{3}

\makeatletter

\makeatother

% Counter untuk figure dan table.
\counterwithin{figure}{chapter}
\counterwithin{table}{chapter}

% Define blank page
\newcommand*{\blankpage}{\afterpage{\null\newpage}}

% Translate autoref into Indonesian
\renewcommand*{\equationautorefname}{Persamaan}%
\renewcommand*{\footnoteautorefname}{catatan kaki}%
\renewcommand*{\itemautorefname}{item}%
\renewcommand*{\figureautorefname}{Gambar}%
\renewcommand*{\tableautorefname}{Tabel}%
\renewcommand*{\partautorefname}{Bagian}%
\renewcommand*{\appendixautorefname}{Lampiran}%
\renewcommand*{\chapterautorefname}{Bab}%
\renewcommand*{\sectionautorefname}{Subbab}%
\renewcommand*{\subsectionautorefname}{Subsubbab}%
\renewcommand*{\subsubsectionautorefname}{Subsubsubbab}%
\renewcommand*{\paragraphautorefname}{paragraf}%
\renewcommand*{\subparagraphautorefname}{subparagraf}%
\renewcommand*{\FancyVerbLineautorefname}{garis}%
\renewcommand*{\theoremautorefname}{Teorema}%
\renewcommand*{\pageautorefname}{halaman}%

% Setting codeblocks
\definecolor{mygreen}{rgb}{0,0.6,0}
\definecolor{mygray}{rgb}{0.5,0.5,0.5}
\definecolor{mymauve}{rgb}{0.58,0,0.82}

\lstset{ 
  backgroundcolor=\color{white},   % choose the background color; you must add \usepackage{color} or \usepackage{xcolor}; should come as last argument
  basicstyle=\ttfamily\footnotesize,        % the size of the fonts that are used for the code
  breakatwhitespace=false,         % sets if automatic breaks should only happen at whitespace
  breaklines=true,                 % sets automatic line breaking
  captionpos=b,                    % sets the caption-position to bottom
  commentstyle=\color{mygreen},    % comment style
  escapeinside={\%*}{*)},          % if you want to add LaTeX within your code
  extendedchars=true,              % lets you use non-ASCII characters; for 8-bits encodings only, does not work with UTF-8
  frame=single,	                   % adds a frame around the code
  keepspaces=true,                 % keeps spaces in text, useful for keeping indentation of code (possibly needs columns=flexible)
  keywordstyle=\color{blue},       % keyword style
  rulecolor=\color{black},         % if not set, the frame-color may be changed on line-breaks within not-black text (e.g. comments (green here))
  showspaces=false,                % show spaces everywhere adding particular underscores; it overrides 'showstringspaces'
  showstringspaces=false,          % underline spaces within strings only
  showtabs=false,                  % show tabs within strings adding particular underscores
  stringstyle=\color{mymauve},     % string literal style
  tabsize=2,	                   % sets default tabsize to 2 spaces
  title=\lstname                   % show the filename of files included with \lstinputlisting; also try caption instead of title
}

% Setting forest
\forestset{default preamble={tikz+={\tikzset{show background rectangle}}}}


\makeatletter

\makeatother

\addbibresource{references.bib}

\begin{document}

%Basic configuration
\title{Pemanfaatan Berbagai Metode \textit{Parameter-Efficient Transfer Learning} pada Kakas Evaluasi IndoLEM}
\date{} 
\author{
    Adiyansa Prasetya Wicaksana \\
    NIM: 13520044
}
\newcommand\tanggalpengesahan{DD MMMM 2024}
\newcommand\PEFT{\textit{parameter-efficient fine-tuning}}
\newcommand\methodPEFT{LoRA, \textit{Prefix-Tuning}, dan \textit{Bottleneck Adapter}}
\newcommand\nlptask{\textit{named entity recognition} (NER), \textit{sentiment analysis}, dan \textit{summarization}}

\pagenumbering{roman}
\setcounter{page}{1}

\input{chapters/cover}
\clearpage
\pagestyle{empty}

\begin{center}
    \smallskip
    
    \Large \bfseries \MakeUppercase{\thetitle}
    \vfill
    
    \Large Laporan Tugas Akhir
    \vfill
    
    \large Oleh
    
    \Large \theauthor
    
    \large Program Studi Teknik Informatika \\
    
    \normalsize \normalfont
    Sekolah Teknik Elektro dan Informatika \\
    Institut Teknologi Bandung \\
    
    \vfill
    \normalsize \normalfont
    Telah disetujui dan disahkan sebagai draft Laporan Tugas Akhir\\ 
    di Bandung, \tanggalpengesahan \\
    Mengetahui,
    
    \vspace{0.5cm}
    Pembimbing,
    
    \vfill
    \underline{Dr. Fariska Zakhralativa Ruskanda, S.T., M.T.} \\
    NIP. 119 110 075
    
\end{center}
\clearpage

% \input{chapters/statement}

\pagestyle{plain}

% \clearpage
\chapter*{ABSTRAK}
\addcontentsline{toc}{chapter}{ABSTRAK}

\begin{center}
    \center
    \begin{singlespace}
        \large\bfseries\MakeUppercase{\thetitle}
    
        \normalfont\normalsize
        Oleh:
    
        \bfseries \theauthor
    \end{singlespace}
\end{center} 

\begin{singlespace}
    Metode \textit{fine-tuning} digunakan sebagai metode pelatihan untuk melakukan evaluasi pada berbagai evaluasi NLU. Tugas evaluasi IndoLEM yang merupakan pionir dari evaluasi NLU berbahasa Indonesia menggunakan \textit{fine-tuning} sebagai metode pelatihannya. \textit{Fine-tuning} melatih model dengan mengubah seluruh parameter model. Hal ini bisa menjadi tantangan dari segi memori dan juga waktu pelatihannya. Terdapat metode \PEFT yang dapat melatih model dengan kinerja yang sebanding dengan metode \textit{fine-tuning}. Dalam tugas akhir ini, berbagai metode \PEFT, yaitu \methodPEFT dimaanfaatkan dalam tugas evaluasi IndoLEM tersebut. Tujuan dari tugas akhir ini adalah untuk memanfaatkan metode PEFT pada IndoLEM. Pemanfaatan tersebut mencakup, penggunaan metode PEFT pada IndoLEM, perbandingkan kinerja terhadap setiap metode PEFT, dan analisis terkait penggunaan parameter dan waktu pelatihan.
    Tugas akhir ini berhasil memanfaatkan metode PEFT pada IndoLEM. Dengan dilakukan refaktorisasi terhadap IndoLEM, metode PEFT dapat dimanfaatkan. Selanjutnya, Eksperimen dilakukan dengan melatih model dengan metode \textit{fine-tuning} dan metode PEFT. Pengujian dilakukan pada 3 tugas evaluasi, yaitu \nlptask. Hasil eksperimen menunjukkan bahwa PEFT hanya menggunakan sekitar 0,2\% sampai 15\% dari parameter pelatihan model, dengan menggunakan waktu pelatihan yang lebih cepat. Kinerja yang didapatkan untuk tugas NER dan \textit{sentiment analysis} berkisar pada rentang -0,8\% sampai -6,2\%. Hal ini menunjukkan adanya \textit{trade off} antara penggunaan parameter pelatihan dengan kinerja yang dihasilkan. Namun, metode \textit{Prefix-Tuning} dan UniPELT gagal untuk memberikan hasil yang konsisten pada tugas \textit{summarization}.

    \textit{\textbf{Kata kunci: Fine-tuning, Parameter-efficient, IndoLEM, NER, Sentiment Analysis, Summarization}}
\end{singlespace}

\clearpage

% \clearpage
\chapter*{ABSTRACT}
\addcontentsline{toc}{chapter}{ABSTRACT}

\begin{center}
    \center
    \begin{singlespace}
        \large\bfseries\MakeUppercase{\titleen}
    
        \normalfont\normalsize
        By:
    
        \bfseries \theauthor
    \end{singlespace}
\end{center} 

\begin{singlespace}
    The fine-tuning method is employed as a training approach for evaluating various NLU tasks. IndoLEM, which is a pioneer in the evaluation of Indonesian-language NLU, uses fine-tuning as its training method. Fine-tuning involves training the model by modifying all of its parameters, which can be challenging in terms of memory and training time. There exists a method called PEFT that can train models with performance comparable to fine-tuning. In this thesis, various PEFT methods, namely \methodPEFT, are utilized in the evaluation tasks of IndoLEM. The aim of this thesis is to leverage PEFT methods in IndoLEM, including the incorporating of PEFT methods, performance comparisons for each PEFT method, and analysis of parameter usage and training time.
    This thesis successfully leverages PEFT methods on IndoLEM. Through refactoring of IndoLEM, PEFT methods were successfully incorporated. Subsequently, experiments were conducted by training models using both fine-tuning and PEFT methods. Testing was carried out on three evaluation tasks, namely \nlptask. The experimental results indicate that PEFT only uses approximately 0.2\% to 15\% of the model's training parameters, with faster training times. The performance achieved for the NER and sentiment analysis tasks ranged from -0.8\% to -6.2\%. This indicates a trade-off between the use of training parameters and the resulting performance. However, the Prefix-Tuning and UniPELT methods failed to provide consistent results on the summarization task.

    \textit{\textbf{Keywords: Fine-tuning, Parameter-efficient, IndoLEM, NER, Sentiment Analysis, Summarization}}
\end{singlespace}

\clearpage

% \chapter*{KATA PENGANTAR}
\addcontentsline{toc}{chapter}{KATA PENGANTAR}

Puji dan syukur penulis panjatkan kepada Tuhan Yang Maha Esa atas berkat dan rahmatnya, laporan tugas akhir yang berjudul \thetitle{} dapat diselesaikan dalam rangka memenuhi syarat kelulusan tingkat sarjana. Perlu diakui pengerjaan tugas akhir ini didukung oleh banyak pihak. Khususnya, penulis ingin mengucapkan terima kasih kepada:

\begin{enumerate}
    \item{
            Ibu Dr. Fariska Zakhralativa Ruskanda, S.T., M.T. selaku dosen pembimbing tugas akhir atas segala bentuk bimbingan dan dukungan yang sangat membantu dalam pengerjaan tugas akhir ini.
        }
    \item{
            Dicky Prima Satya, S.T, M.T., Bapak Adi Mulyanto, S.T, M.T., Robithoh Annur, S.T., M.Eng., Ph.D., dan Tricya Esterina Widagdo, ST., M.Sc selaku dosen koordinator tim tugas akhir atas segala bentuk materi dan usahanya untuk mempermudah mahasiswa dalam pengerjaan tugas akhir. 
        }
    \item{
            Seluruh dosen program studi Teknik Informatika ITB atas segala bentuk ilmu yang telah diajarkan kepada mahasiswa. Semoga seluruh ilmu yang diajarkan bisa terus bermanfaat.
        }
    \item{
            Seluruh \textit{staff} fakultas STEI atas segala jasanya dalam memudahkan mahasiswa pada proses perkuliahan.
        }
    \item{
            Bapak Ade Hermawan dan Ibu Annik Oktaelly selaku orang tua atas segala doa dan bantuan yang bisa diberikan dalam pengerjaan tugas akhir ini. Anggun dan Dean, selaku saudara dari penulis atas segala dukungannya. Semoga semua hal baik selalu menyertai mereka.
        }
    \item{
            Sahabat terbaik, Trista, yang selalu menemani dan memberikan dukungan yang terbaik setiap saatnya.
        }
    \item{
            Sahabat terdekat dari SMA, Haffif, Samuel, Kiagus, Patria, Aji, Adit, Yoga, Mak, Rafy, Revan, Ocit, Reyhan, Vigar, Aysar, Hanip dan sahabat-sahabat lainnya dari SMA atas seluruh memori dan pelajaran hidup yang diberikan.
        }
    \item{
            Sahabat terdekat dari perkuliahan, Gagas, Dhika, Gare, Kinan, Marcho, Aira, Dipa, Januar, Ubai, Fikri, Fikron, Kibar, Eja, Azka, Azka, Epi, Ken, Arik, Fikri, Kevin, Ocep, Shadiq, Jova, Saul, Chus, Hanip, Kosar, Adila, dan teman-teman lainnya dari perkuliahan atas seluruh kenangan, obrolan, dan pelajaran hidup yang diberikan.
        }
    \item{
            Seluruh pihak lainnya yang tidak bisa disebutkan yang secara tidak langsung ataupun langsung membantu pengerjaan tugas akhir ini.
        }
\end{enumerate}

Akhir kata, penulis mengucapkan terima kasih kepada semua pihak yang telah terlibat dalam pengerjaan tugas akhir ini. Penulis juga ingin menyampaikan mohon maaf apabila terdapat kesalahan maupun kekurangan dalam laporan tugas akhir ini. Penulis berharap semoga tugas akhir ini dapat bermanfaat bagi pembaca dan riset-riset ke depannya.

\begin{flushright}
    \vspace{0.5cm}
    Bandung, \tanggalpengesahan


    \vspace{1.5cm}

    Adiyansa Prasetya Wicaksana
\end{flushright}


\titlespacing*{\chapter}{0pt}{0pt}{4pt}

% Setting judul toc, lot, lof, bib
\renewcommand{\contentsname}{DAFTAR ISI}
\renewcommand{\listfigurename}{DAFTAR GAMBAR}
\renewcommand{\listtablename}{DAFTAR TABEL}
\renewcommand{\bibname}{DAFTAR PUSTAKA}

\tableofcontents
% \listofappendices
\listoffigures
\listoftables
\newpage
\begin{center}
    \Large \bfseries{DAFTAR ISTILAH}
    \bigskip

    \begin{table}[h]
        \begin{tabularx}{\textwidth}{lcX}
            \textbf{NLP} & : & \textit{Natural language processing} / pengolahan bahasa alami \\
            \textbf{NLU} & : & \textit{Natural language understanding} \\
            \textbf{NLP} & : & \textit{Natural language generation} \\
            \textbf{GLUE} & : & \textit{General language understanding evaluation} \\
            \textbf{NER} & : & \textit{Named entity recognition} \\
            \textbf{PEFT} & : & \textit{Parameter-efficient fine-tuning} \\
            \textbf{LoRA} & : & \textit{Low rank adaptation} \\
        \end{tabularx}
    \end{table}
\end{center}


\newpage

\titleformat*{\section}{\bfseries\large}
\pagenumbering{arabic}

%----------------------------------------------------------------%
% Konfigurasi Bab
%----------------------------------------------------------------%
\setcounter{page}{1}
\renewcommand{\chaptername}{BAB}
\renewcommand{\thechapter}{\Roman{chapter}}
%----------------------------------------------------------------%

%----------------------------------------------------------------%
% Dafter Bab
% Untuk menambahkan daftar bab, buat berkas bab misalnya `chapter-6` di direktori `chapters`, dan masukkan ke sini.
%----------------------------------------------------------------%
\chapter{Pendahuluan}

Bab ini berisi terkait gambaran umum dan permasalahan yang akan diselesaikan dalam tugas akhir ini. Bab ini akan dimulai dari penjelasan latar belakang dari masalah yang diselesaikan, rumusan masalah, tujuan, batasan masalah, metodologi yang digunakan, dan berakhir pada sistematika penulisan tugas akhir ini.

\section{Latar Belakang}

Pada era digital saat ini, pengolahan bahasa alami (\textit{Natural Language Processing}, NLP) telah menjadi salah satu bidang yang sangat penting dan mengalami perkembangan pesat. NLP memungkinkan komputer untuk memahami, menginterpretasi, dan merespons bahasa manusia dengan cara yang bermakna. Salah satu model yang telah terbukti efektif dalam berbagai aplikasi NLP adalah BERT (\textit{Bidirectional Encoder Representations from Transformers}). IndoBERT, sebagai variasi dari model BERT yang dirancang khusus untuk bahasa Indonesia, telah menjadi pilihan utama dalam pengolahan bahasa alami berbahasa Indonesia.

Meskipun IndoBERT telah memberikan hasil yang baik dalam berbagai tugas NLP, masih ada potensi untuk meningkatkan kinerjanya lebih lanjut. Salah satu pendekatan yang muncul untuk meningkatkan kinerja model bahasa seperti IndoBERT adalah melalui penggunaan teknik \textit{parameter-efficient transfer learning}. Teknik ini memungkinkan model untuk mempelajari tugas-tugas spesifik dengan jumlah parameter yang lebih sedikit, yang pada gilirannya dapat meningkatkan efisiensi penggunaan sumber daya, terutama parameter model.

Dalam konteks ini, teknik \textit{parameter-efficient transfer learning} memiliki berbagai variasi yang menarik, seperti LoRA (\textit{Low-Rank Adaptation}), \textit{Prefix-Tuning}, \textit{Tiny-Attention Adapter}, dan \textit{Unified View of Parameter-Efficient Transfer Learning}. Setiap teknik memiliki karakteristiknya sendiri, dan hingga saat ini, belum ada banyak penelitian yang secara komprehensif membandingkan kinerja antara teknik-teknik ini, terutama ketika diterapkan pada model IndoBERT.

Penelitian ini akan difokuskan pada penerapan berbagai teknik \textit{parameter-efficient transfer learning} pada model IndoBERT dengan tujuan mengukur sejauh mana teknik-teknik ini dapat meningkatkan kinerja model dalam berbagai tugas NLP. Selain itu, penelitian ini akan membandingkan kinerja berbagai teknik \textit{parameter-efficient transfer learning} dengan metode \textit{fine-tuning} tradisional. Fokus utama penelitian ini adalah bagaimana efisiensi penggunaan sumber daya, terutama parameter model, dapat mempengaruhi peningkatan kinerja model bahasa IndoBERT.

Dengan demikian, penelitian ini diharapkan dapat memberikan kontribusi dalam pengembangan metode untuk meningkatkan kinerja model bahasa IndoBERT, sekaligus memberikan pemahaman yang lebih baik mengenai efektivitas dan efisiensi berbagai teknik \textit{parameter-efficient transfer learning} dalam konteks bahasa Indonesia.
\section{Rumusan Masalah}

Dalam konteks pengembangan IndoBERT, penelitian ini akan menjawab beberapa pertanyaan utama, yaitu:

\begin{enumerate}
    \item Apakah berbagai metode \PEFT, yaitu \methodPEFT, dapat diimplementasikan pada kakas evaluasi IndoLEM?
    \item Apakah ada perbedaan signifikan dalam kinerja antara teknik-teknik \PEFT tersebut?
    \item Apakah waktu dan penggunaan parameter pada proses pelatihan dengan metode \PEFT lebih efisien dibandingkan dengan \textit{fine-tuning} tradisional?
\end{enumerate}

\section{Tujuan}

Tujuan yang akan dicapai untuk tugas akhir ini adalah sebagai berikut.

\begin{enumerate}
    \item Meningkatkan kinerja model pada hasil evaluasi IndoLEM dengan berbagai metode \PEFT.
    \item Membandingkan teknik \PEFT, yaitu \methodPEFT dalam meningkatkan kinerja model.
    \item Mengevaluasi waktu dan penggunaan parameter pada proses pelatihan antara metode \PEFT dengan \textit{fine-tuning} tradisional.
\end{enumerate}

\section{Batasan Masalah}
\label{sec:batasan-masalah}

Terdapat batasan yang diambil dalam pelaksanaan tugas akhir ini, yaitu sebagai berikut.

\begin{enumerate}
    \item Penelitian ini akan berfokus pada peningkatan kinerja model bahasa IndoBERT dalam berbagai tugas bahasa, seperti \textit{text classification},  \textit{named entity recognition} (NER), dan \textit{sentiment analysis}.
    \item Eksperimen akan menggunakan dataset yang telah ada dan tersedia untuk berbagai tugas bahasa, tanpa menghasilkan dataset baru.
    \item Evaluasi kinerja model-model yang telah di-\textit{fine-tuning} akan menggunakan IndoLEM.
    \item Analisis efisiensi akan mempertimbangkan penggunaan sumber daya, terutama jumlah parameter model, sebagai salah satu faktor dalam pemilihan teknik \textit{transfer learning} yang efisien.
    \item Penelitian ini akan dilakukan dalam konteks bahasa Indonesia dan tidak akan mencakup pengembangan model bahasa untuk bahasa lain.
\end{enumerate}

\section{Metodologi}

Terdapat metodologi yang digunakan untuk melaksanakan tugas akhir ini, berikut adalah tahapan pelaksanaan.

\begin{enumerate}
    \item Pengumpulan dataset dan persiapan data untuk berbagai tugas bahasa yang akan digunakan dalam eksperimen.
    \item \textit{Pre-training} model IndoBERT pada korpus teks bahasa Indonesia yang besar.
    \item Implementasi berbagai teknik \textit{transfer learning}, seperti LoRA, \textit{Prefix-Tuning}, \textit{Tiny-Attention Adapter}, dan \textit{Unified Parameter Transfer Learning}, pada model IndoBERT yang telah di-\textit{pretrain}.
    \item \textit{Fine-tuning} model-model hasil \textit{transfer learning} pada berbagai tugas bahasa yang telah disiapkan.
    \item Evaluasi kinerja model-model yang telah di-\textit{fine-tuning} menggunakan metrik yang sesuai untuk masing-masing tugas.
    \item Perbandingan hasil dan efisiensi penggunaan sumber daya antara teknik \textit{transfer learning} dan \textit{fine-tuning} tradisional.
\end{enumerate}
\section{Sistematika Pembahasan}

Konten dari Tugas Akhir ini  dibagi menjadi lima bab sebagai berikut.
\begin{enumerate}
    \item Pendahuluan

    Pada Bab I  dijelaskan gagasan utama dari tugas akhir ini yang berisi dari latar belakang, rumusan masalah, tujuan, batasan, metodologi hingga sistematika pembahasan.

    \item Studi Literatur

    Selanjutnya, Bab II  menjelaskan hasil studi literatur yang berkaitan dengan pengerjaan tugas akhir ini. Bab II ini berisi tentang pemahaman dasar seputar topik yang  dibahas pada tugas akhir ini.

    \item Analisis Persoalan dan Rancangan Solusi

    Pada Bab III  dijelaskan analisis persoalan untuk menyusun rancangan solusi. Rancangan tersebut  dijelaskan pada bab ini sebelum diimplementasikan. Rancangan solusi  dipaparkan dalam bentuk diagram dan kajian dalam bab ini.

    \item Implementasi dan Pengujian

    Bab IV ini  berisikan kajian terhadap implementasi yang telah dibuat. Bab ini juga  membahas tahap-tahap pengujian dan hasilnya. Perbandingan beberapa model prediksi  dibahas pada bab ini.

    \item Kesimpulan dan Saran

    Bab V  menutup tugas akhir ini. Konten pada bab ini  menjawab rumusan masalah. Bab ini juga  menyebutkan saran-saran perbaikan yang bisa dipakai untuk penelitian berikutnya. Bab ini  menyimpulkan hasil implementasi dan rancangan solusi terhadap masalah yang sudah diidentifikasi.
\end{enumerate}


\chapter{Studi Literatur}

Pada bab ini, akan diisi oleh studi literatur, 
hal-hal yang berkaitan dengan topik persoalan tugas akhir 
akan dipaparkan dalam bab ini guna untuk memberikan informasi 
mengenai dasar teori dan studi yang dipakai. 
Bab ini diharapkan membantu pembaca untuk mengerti 
dalam membaca penelitian tugas akhir ini.

\section{\textit{Natural Language Processing}}

Pemrosesan Bahasa Alami (PBA) atau dalam bahasa Inggris dikenal dengan \textit{Natural Language Processing} (NLP) merupakan cabang dari ilmu komputer, kecerdasan buatan, dan linguistik yang berfokus pada interaksi antara komputer dan bahasa manusia. NLP bertujuan untuk memungkinkan komputer tidak hanya memahami dan menafsirkan bahasa manusia, tetapi juga untuk menghasilkannya dengan cara yang bermakna dan efektif. Hal ini dijelaskan oleh \citeauthor{nlp} \parencite{nlp}, menyatakan pentingnya NLP dalam membangun jembatan komunikasi antara manusia dan mesin.

Dalam beberapa dekade terakhir, NLP telah mengalami kemajuan yang signifikan, memungkinkan komputer tidak hanya memahami bahasa manusia tetapi juga merespons dengan cara yang semakin kompleks dan kontekstual. Teknologi seperti mesin penerjemah, asisten virtual, dan sistem rekomendasi semuanya memanfaatkan prinsip-prinsip NLP untuk berfungsi.

Salah satu tantangan utama dalam NLP adalah keragaman dan kompleksitas bahasa manusia. Bahasa penuh dengan nuansa, ambiguitas, dan struktur yang dapat bervariasi tergantung pada konteks dan budaya. Untuk mengatasi tantangan ini, berbagai tugas NLP telah didefinisikan dan dikembangkan untuk memecah masalah pemahaman bahasa menjadi komponen yang lebih kecil dan lebih spesifik.

Beberapa tugas NLP yang umum antara lain \textit{Part of Speech} (POS) \textit{Tagging}, \textit{Named Entity Recognition} (NER), \textit{Dependency Parsing}, \textit{Sentiment Analysis}, dan \textit{Summarization}. Masing-masing tugas ini menargetkan aspek tertentu dari pemahaman bahasa dan memiliki aplikasi praktisnya sendiri dalam berbagai bidang, mulai dari analisis teks hingga pengembangan sistem percakapan otomatis.

\section{\textit{Transformer}}

\textit{Transformer} telah mengubah lanskap pemrosesan bahasa alami (NLP) dengan cara yang signifikan. Dalam penelitian mereka, penulis memperkenalkan konsep baru yang disebut mekanisme \textit{self-attention} \parencite{transformers}. Mekanisme ini memungkinkan setiap kata dalam input untuk memfokuskan pada kata-kata lain dalam sekuens yang sama, memberikan model kemampuan untuk memahami konteks dengan lebih baik. Ini berbeda dari pendekatan yang biasanya mengandalkan informasi lokal atau posisi tetap dalam sekuens.

\begin{figure}[ht]
    \centering
    \includegraphics[width=0.8\textwidth]{chapter-2/transformer.png}
    \caption{Arsitektur \textit{Transformer} \parencite{transformers}}
    \label{fig:transformer}
\end{figure}

Arsitektur dari \textit{transformer} dapat dilihat pada Gambar \ref{fig:transformer}. Salah satu keunggulan utama dari mekanisme \textit{self-attention} adalah kemampuannya untuk menangani sekuens dengan panjang yang berbeda dan memahami hubungan antar kata tanpa mempertimbangkan jarak antara mereka. Ini memungkinkan \textit{transformer} untuk memahami ketergantungan jarak jauh dalam teks, sesuatu yang sulit dicapai oleh arsitektur sebelumnya seperti RNN dan LSTM.

Selain itu, \textit{transformer} dirancang untuk paralelisasi, yang memungkinkannya dilatih dengan cepat pada perangkat keras modern. Ini mempercepat penelitian dan pengembangan dalam NLP dan memungkinkan \textit{training} model skala besar seperti BERT dan GPT. Sejak diperkenalkannya \textit{transformer}, banyak variasi dan peningkatan telah dikembangkan. Namun, prinsip dasar \textit{self-attention} dan paralelisasi yang diperkenalkan oleh \textit{transformer} tetap menjadi inti dari banyak inovasi dalam NLP.


\section{\textit{Bidirectional Encoder Representations from Transformers} \\ (BERT)}

BERT, yang merupakan singkatan dari \textit{Bidirectional Encoder Representations from Transformers}, adalah model pemrosesan bahasa alami yang diperkenalkan oleh Google pada tahun 2018 \parencite{bert}. BERT memanfaatkan arsitektur \textit{transformer}, yang telah dibahas sebelumnya, untuk memahami konteks kata dalam teks dengan cara yang lebih mendalam daripada pendekatan sebelumnya.

Salah satu keunggulan utama BERT adalah pendekatannya yang \textit{bidirectional}. Sebagai gantinya dari hanya memahami teks dari kiri ke kanan atau sebaliknya, BERT memahami konteks kata dengan mempertimbangkan informasi dari kedua arah. Ini memungkinkan model untuk memiliki pemahaman yang lebih kaya tentang makna dan nuansa dalam teks.

BERT telah dilatih pada sejumlah besar teks, yang memungkinkannya untuk mengembangkan representasi kata yang kaya dan mendalam. Ketika digunakan untuk tugas-tugas NLP spesifik, seperti klasifikasi teks atau pemahaman pertanyaan, BERT dapat disesuaikan dengan data tugas spesifik untuk mencapai kinerja yang luar biasa.

\subsection{IndoBERT}

IndoBERT adalah adaptasi dari model BERT yang khusus dilatih untuk bahasa Indonesia. Mengingat keunikan dan kompleksitas bahasa Indonesia, memiliki model yang khusus dilatih untuk bahasa ini sangat penting untuk memastikan kinerja yang optimal pada tugas-tugas NLP yang berfokus pada bahasa Indonesia.

IndoBERT memanfaatkan kekuatan arsitektur BERT sambil disesuaikan dengan karakteristik dan nuansa bahasa Indonesia. Ini memungkinkan model untuk menangkap makna, idiom, dan struktur bahasa dengan lebih akurat, menjadikannya alat yang sangat berharga untuk peneliti dan praktisi yang bekerja dengan teks berbahasa Indonesia.

\section{\textit{Transfer Learning}}

\textit{Transfer Learning} merupakan salah satu pendekatan kunci dalam pembelajaran mesin yang memanfaatkan model yang telah dilatih pada tugas tertentu sebagai dasar untuk melatih model pada tugas lain. Ide dasar di balik \textit{transfer learning} adalah bahwa, jika model telah mempelajari fitur-fitur tertentu dari satu tugas, fitur-fitur tersebut dapat digunakan sebagai informasi awal yang berguna untuk tugas lain.

Sebagai contoh, model yang telah dilatih untuk mengenali objek dalam gambar dapat memanfaatkan pengetahuannya tentang fitur visual, seperti tepi atau tekstur, saat dilatih untuk tugas pengenalan wajah. Meskipun tugas awal (mengenali objek) dan tugas kedua (pengenalan wajah) berbeda, ada sejumlah fitur visual yang relevan untuk kedua tugas tersebut.

Dalam konteks pemrosesan bahasa alami (NLP), \textit{transfer learning} sering digunakan untuk memanfaatkan \textit{Large Language Model} (LLM) yang telah dilatih pada korpus teks besar untuk tugas-tugas spesifik seperti \textit{sentiment analysis} atau \textit{named entity recognition}. Dengan memulai dari model yang telah memiliki pemahaman dasar tentang struktur dan semantik bahasa, proses pelatihan untuk tugas spesifik menjadi lebih cepat dan seringkali menghasilkan model yang lebih akurat dibandingkan dengan melatih model dari awal.

Keuntungan lain dari \textit{transfer learning} adalah efisiensi komputasi. Melatih model pembelajaran mesin dari awal, terutama model dengan banyak parameter, memerlukan sumber daya komputasi yang signifikan. Dengan menggunakan model yang telah dilatih sebagai titik awal, dapat menghemat waktu dan sumber daya komputasi, sambil mempertahankan atau bahkan meningkatkan kinerja model.


\subsection{\textit{Low Rank Adaptation} (LoRA)}

Dalam dunia pembelajaran mesin, terutama saat bekerja dengan model berukuran besar, efisiensi parameter menjadi salah satu tantangan utama. LoRA, singkatan dari \textit{Low Rank Adaptation}, muncul sebagai solusi untuk tantangan ini dalam konteks \textit{transfer learning}.

Konsep dasar di balik LoRA adalah ide bahwa adaptasi model untuk tugas baru tidak selalu memerlukan perubahan besar pada seluruh parameter model. Sebaliknya, perubahan kecil pada representasi tertentu dapat menghasilkan peningkatan kinerja yang signifikan. Dengan fokus pada \textit{low rank adaptation}, LoRA mengubah hanya sebagian kecil dari bobot model, sementara sebagian besar bobot lainnya tetap tidak berubah. Ini berarti bahwa hanya "sebagian" dari informasi dalam model yang diperbarui, yang mengarah pada efisiensi komputasi yang meningkat \parencite{lora}.

Salah satu kelebihan utama dari pendekatan ini adalah kemampuannya untuk mengurangi \textit{overhead} komputasi. Dalam praktiknya, ini berarti bahwa waktu pelatihan dan sumber daya yang diperlukan untuk adaptasi model menjadi jauh lebih sedikit dibandingkan dengan metode lain yang mungkin melibatkan pelatihan ulang model dari awal atau menambahkan sejumlah besar parameter tambahan.

Pendekatan LoRA menjadi sangat relevan dan berharga, terutama saat berhadapan dengan model-model berukuran besar seperti GPT-3. Model-model seperti ini memiliki jumlah parameter yang sangat besar, sehingga pelatihan ulang atau menambahkan parameter tambahan bisa menjadi sangat mahal dari segi komputasi. Dengan LoRA, adaptasi model-model besar menjadi lebih praktis dan dapat dilakukan dengan efisiensi yang jauh lebih tinggi, tanpa mengorbankan kinerja.

Dengan demikian, LoRA menawarkan pendekatan yang menjanjikan untuk mengadaptasi \textit{pre-trained model} dengan cara yang lebih efisien, memungkinkan peneliti dan praktisi untuk memanfaatkan kekuatan model berukuran besar tanpa harus berurusan dengan beban komputasi yang berat.
\subsection{\textit{Tiny Attention Adapter}}

\textit{Tiny-Attention Adapter} merupakan salah satu solusi yang dirancang untuk mengatasi komputasi. Sebagai gantinya dari menambahkan lapisan adaptasi berukuran besar atau melatih ulang seluruh model, teknik ini memperkenalkan konsep "tiny-attention" \parencite{tinyattention}. Mekanisme ini, meskipun sederhana, memungkinkan setiap posisi dalam sekuens untuk memperhatikan dan memodifikasi keadaan tersembunyinya berdasarkan informasi dari semua posisi lain dalam sekuens. Dengan kata lain, setiap elemen dalam sekuens memiliki kemampuan untuk "berkomunikasi" dan "berkoordinasi" dengan elemen lain untuk membentuk representasi yang lebih kaya dan kontekstual.

Salah satu kelebihan dari pendekatan ini adalah fleksibilitas dan dinamikanya. Karena setiap posisi dapat memperhatikan semua posisi lain, model memiliki kapasitas untuk memahami hubungan antar kata dengan lebih baik, terutama hubungan yang bersifat jarak jauh atau kontekstual yang kompleks. Ini memungkinkan model untuk menangkap nuansa dan makna yang mungkin terlewatkan oleh teknik adaptasi lainnya.

Meskipun pendekatan ini terfokus pada efisiensi parameter, \textit{Tiny-Attention Adapter} tidak mengorbankan kinerja. Sebaliknya, berkat mekanisme "tiny-attention", teknik ini seringkali mampu mencapai kinerja yang sebanding atau bahkan melampaui metode adaptasi tradisional, meskipun hanya dengan sebagian kecil dari parameter tambahan.

Dengan demikian, \textit{Tiny-Attention Adapter} menawarkan pendekatan yang menjanjikan untuk mengadaptasi model pra-latih dengan cara yang lebih efisien, tanpa mengorbankan kualitas atau kinerja.
\subsection{\textit{Unified View Parameter Learning}}

\textit{Unified Parameter Transfer Learning} merupakan strategi canggih dalam \textit{transfer learning} yang berupaya menggabungkan berbagai teknik adaptasi ke dalam satu kerangka kerja terpadu \parencite{uvpl}. Tujuan utamanya adalah untuk memaksimalkan efisiensi dan efektivitas saat mengadaptasi model pra-latih untuk tugas-tugas baru.

Dalam \textit{transfer learning}, ide utamanya adalah mengambil model yang telah dilatih pada satu tugas dan menyesuaikannya untuk tugas yang berbeda. Namun, ada banyak cara untuk melakukan adaptasi ini, dan setiap metode memiliki kelebihan dan kekurangannya sendiri. Beberapa teknik mungkin fokus pada penambahan lapisan adaptasi, sementara yang lain mungkin memprioritaskan modifikasi parameter tertentu dalam model.

Alih-alih memilih satu teknik adaptasi dan berkomitmen padanya, pendekatan ini menggabungkan berbagai teknik untuk menciptakan solusi adaptasi yang lebih komprehensif. Dengan demikian, model yang diadaptasi dengan metode ini dapat memanfaatkan kelebihan dari berbagai teknik adaptasi, sambil menghindari atau meminimalkan kekurangan masing-masing teknik.

Misalnya, satu teknik adaptasi mungkin sangat efektif untuk tugas klasifikasi tetapi kurang optimal untuk tugas generasi teks. Dengan pendekatan terpadu, model dapat memanfaatkan teknik adaptasi yang paling sesuai untuk setiap tugas, memungkinkan kinerja yang lebih baik dan adaptasi yang lebih cepat.

Selain itu, dengan menggabungkan berbagai teknik, \textit{Unified Parameter Transfer Learning} juga memungkinkan peneliti untuk bereksperimen dan menemukan kombinasi teknik yang paling efektif untuk tugas atau dataset tertentu. Ini memberikan fleksibilitas tambahan dan memungkinkan adaptasi yang lebih disesuaikan dengan kebutuhan spesifik.

Dengan demikian, \textit{Unified Parameter Transfer Learning} menawarkan pendekatan adaptasi yang inovatif dan fleksibel, yang memungkinkan model pra-latih untuk menyesuaikan diri dengan berbagai tugas dengan efisiensi dan efektivitas yang lebih tinggi.
% \blankpage
\chapter{Analisis Persoalan dan Rancangan Solusi}
\label{sec:chapter-3}

Tujuan utama penulisan bab ini adalah untuk menguraikan rencana penyelesaian masalah tugas akhir sebelum dieksekusi. Bagian ini  memaparkan proses analisis masalah hingga menjadi solusi.

\section{Analisis Persoalan}
\label{sec:analisis-persoalan}

Berdasarkan latar belakang yang telah diuraikan pada subbab \ref{sec:latar-belakang}, peningkatan kinerja model IndoBERT dilakukan dengan menggunkan metode \PEFT (PEFT). Peningkatan kinerja model dengan metode PEFT ini melibatkan penambahan konfigurasi ataupun \textit{layer} tambahan terhadap model tergantung dengan karateristik dari metode PEFT masing-masing. Selain itu, analisis komprehensif dibutuhkan untuk membandingkan metode PEFT yang dapat memberikan hasil terbaik. Untuk membuat analisis yang komprehensif dibutuhkan metode evaluasi terhadap kinerja model. Secara keseluruhan, penelitian yang dilakukan pada tugas akhir ini akan mencakup tiga tahap sebagai berikut.

\begin{enumerate}
    \item Pengembangan setiap metode PEFT.
    \item Konfigurasi setiap metode PEFT.
    \item Eksperimen dan evaluasi kinerja model.
\end{enumerate}

Berdasarkan tahapan-tahapan penelitian yang diuraikan tersebut, terdapat lima persoalan utama dengan rincian sebagai berikut.

\begin{enumerate}
    \item Kakas IndoLEM yang \textit{outdated}.
    
    Berdasarkan tahap penelitian nomor 1, eksperimen yang dilakukan pada kakas IndoLEM menggunakan versi Python yang lama. Selain itu, pustaka yang digunakan seperti Torch, Transformers, dan Seqeval yang digunakan mempunyai versi yang lama juga. Sehingga, untuk mengembangkan metode PEFT pada kakas IndoLEM harus menggunakan versi yang sesuai dengan IndoLEM-nya. Selain itu, setiap eksperimen pada setiap tugas NLP mempunyai proses pemanggilan \textit{script} yang berbeda antara satu dengan yang lain pada kakas IndoLEM. Padahal, beberapa prosesnya dapat disamakan sehingga akan mempermudah untuk dilakukan pengembangan. Jadi, perlu dilakukan refaktorisasi pada kakas IndoLEM untuk memperbarui versi dan juga mempermudah proses pengembangan. 

    \item Pengembangan metode PEFT pada kakas IndoLEM.
    
    IndoLEM merupakan kakas untuk melakukan training pada model IndoBERT. Berdasarkan tahap penelitian nomor 1, diperlukan pengembangan setiap metode pada kakas IndoLEM. Pengembangan ini mencakup penambahan dan perubahan kode pada kakas tersebut. Setiap metode PEFT mempunyai cara implementasi yang berbeda, sehingga perlu dikembangkan untuk setiap metode yang akan digunakan, yaitu \methodPEFT.

    \item Pemilihan konfigurasi terbaik untuk setiap metode PEFT.
    
    Setiap metode PEFT mempunyai karakteristik yang berbeda dalam penggunaannya, sehingga konfigurasi untuk setiap metode perlu diperhatikan. Proses ini melibatkan penyesuaian \textit{layer} dan parameter yang spesifik untuk setiap metode. Penting untuk mengeksplorasi berbagai konfigurasi untuk menemukan kombinasi yang paling efektif. Hal ini mencakup penyesuaian ukuran \textit{layer}, jumlah parameter, dan aspek teknis lainnya yang dapat mempengaruhi kinerja model. Selain itu, metode PEFT juga memerlukan penambahan \textit{adapter} seperti pada teknik \textit{Bottleneck Adapter}. Bahkan, menggabungkan beberapa metode PEFT juga memungkinkan, sehingga diperlukan pembuatan konfigurasi yang spesifik untuk setiap metode.

    \item Pemilihan kombinasi \textit{hyperparameter} terbaik pada proses \textit{training}.
    
    Berdasarkan tahap penelitian nomor 3 yaitu eksperimen dan evaluasi kinerja model, proses \textit{transfer learning} melibatkan proses \textit{training}. Pada proses tersebut perlu pemilihan kombinasi \textit{hyperparameter} yang optimal. \textit{Hyperparameter} seperti \textit{learning rate}, \textit{batch size}, dan jumlah \textit{epoch} berperan penting dalam menentukan kinerja model. Proses ini melibatkan eksperimen dengan berbagai kombinasi untuk menemukan setelan yang memberikan hasil terbaik. 
    
    \item Pemilihan metode evaluasi.
    
    Untuk dapat melakukan analisis terhadap model diperlukan adanya metode evaluasi yang memang relevan dengan tugas NLP-nya. Metode evaluasi yang digunakan pada proses \textit{training} berupa metrik seperti akurasi, \textit{presisi}, \textit{recall}, dan F1-\textit{score}. Sedangkan, untuk analisis komprehensif yang dilakukan terakhir akan digunakan IndoLEM sebagai \textit{NLP Task Benchmarking}.

\end{enumerate}

\section{Analisis Solusi}

Berdasarkan analisis masalah, terdapat beberapa solusi yang dapat langsung menjawab permasalahan yang diuraikan pada Bab \ref{sec:analisis-persoalan}. Diajukan solusi-solusi yang dijabarkan pada analisis solusi berikut. Setiap bagian dari analisis solusi ini  menjawab setiap bagian pada analisis persoalan.

\subsection{Kakas IndoLEM yang \textit{Outdated}}

Untuk menjawab persoalaan ini, diperlukan pengembangan ulang untuk memperbarui dan memperbaiki infrastruktur dari kakas IndoLEM. Selain itu, proses ini juga diperlukan untuk mengembangkan metode PEFT karena metode tersebut memerlukan versi yang lebih baru dari pustaka yang digunakan. Untuk melakukan pengembangan ulang, terdapat beberapa langkah yang diperlukan, yaitu memperbarui versi dari perangkat lunak yang digunakan, penyederhanaan dan standardisasi proses, serta dokumentasi.

Peningkatan versi perangkat lunak yaitu Python ke versi lebih baru yang kompatibel dengan pustaka yang  digunakan beserta dependensinya, salah satunya adalah Torch dan Transformers. Hal ini perlu dilakukan karena versi pustaka yang digunakan pada kakas IndoLEM banyak yang tidak bisa digunakan pada versi Python yang lebih baru.

\textit{Script} yang saat ini digunakan pada kakas IndoLEM cukup berbeda antar setiap tugas evaluasi. Padahal, untuk setiap tugas evaluasi, proses pelatihannya itu bisa distandardisasi. Dengan menggunakan modul Trainer dari pustaka Transformers, bagian pelatihan bisa distandardisasi. Hanya perlu menyesuaikan pada bagian praproses data dan perhitungan metriksnya.

Dokumentasi pada kakas IndoLEM saat ini cukup terbatas, terutama pada \textit{requirement} untuk versi Python dan pustaka yang digunakannya. Dengan menambahkan \textit{requirement} yang dibutuhkan untuk menajalankan proses pelatihan dan evaluasi dapat menjawab masalah ini.

\subsection{Implementasi Metode PEFT pada Kakas IndoLEM}

Metode PEFT yaitu \methodPEFT perlu diimplementasikan pada kakas IndoLEM. Implementasi dari setiap metode PEFT ini terdapat pada penelitian terkaitnya. Tetapi, sudah banyak kakas yang memungkinkan untuk menggunakan metode ini dengan mengintegrasikannya pada model. Hal ini dilakukan dengan menggunakan pustaka dari metode PEFT tersebut.

Terdapat beberapa pustaka yang sudah mengimplementasikan metode PEFT, ada yang hanya mengimplementasikan satu, beberapa, atau bahkan menggabungkan beberapa metode PEFT tersebut. Salah satunya adalah pustaka PEFT, OpenDelta, dan Adapters. Setiap pustaka tersebut mempunyai metode PEFT yang berbeda. Pustaka PEFT hanya mengimplementasikan metodenya saja tanpa ada fungsionalitas untuk menggabungkan beberapa metode PEFT. Sedangkan, OpenDelta dan Adapters mampu untuk melakukan penggubangan metode PEFT. Hanya saja, metode PEFT yang bisa digunakan pada pustaka OpenDelta dan Adapters lebih sedikit dibandingkan dengan pustaka PEFT.

Pada tugas akhir ini,  dilakukan implementasi pada metode PEFT yaitu \methodPEFT. Selain itu juga, dilakukan penggabungan antara ketiga metode tersebut. Sehingga, yang bisa menjadi pilihan adalah pustaka OpenDelta dan Adapters. Berdasarkan aktivitas dari \textit{soure code}-nya, pustaka Adapters, masih banyak dilakukan perubahan, sedangkan OpenDelta tidak ada perubahan dari sekitar setahun yang lalu. Sehingg, pustaka Adapters yang  digunakan untuk implementasi metode PEFT pada kakas IndoLEM.

\subsection{Pemilihan Model}

Pada penelitian yang menerbitkan kakas IndoLEM, diterbitkan juga sekaligus modelnya yaitu IndoBERT. Seperti yang disebtukan pada subbab \ref{sec:indobet}, model IndoBERT menjadi \textit{state-of-the-art} pada hasil evaluasi pada kakas IndoLEM. IndoBERT merupakan model \textit{encoder} yang cocok untuk tugas \textit{classification}, sehingga sesuai dengan tugas evaluasi NER dan \textit{sentiment analysis}. Namun, untuk tugas \textit{generation} yaitu \textit{summarization} yang membutuhkan model \textit{encoder decoder} bukan hal yang sesuai.

Tugas evaluasi \textit{summarization} pada kakas IndoLEM menggunakan model IndoBERT yang merupakan model \textit{encoder}. Hal ini bisa dilakukan dengan menjadikan model yang sama sebagai \textit{decoder}-nya juga, menjadikan 2 model IndoBERT sebagai \textit{encoder} dan \textit{decoder}. Terdapat model yang merupakan \textit{encoder decoder} yang lebih cocok untuk tugas evaluasi \textit{summarization} yaitu model BART dan T5. Untuk versi bahasa Indonesianya, terdapat model IndoBART dan IndoT5. Berdasarkan hasil evaluasi dari IndoBART dan IndoT5, model IndoT5 menghasilkan evaluasi yang lebih baik daripada IndoBART. 

\subsection{Pelatihan dan Evaluasi Model}

Terdapat beberapa komponen dalam pelatihan dan evaluasi model, yaitu \textit{dataset} (latih, evaluasi dan uji), \textit{hyperparameter} yang digunakan, dan lingkungan pelatihan. Pada kakas IndoLEM, untuk tugas evaluasi \nlptask sudah terdapat \textit{dataset} yang tersedia. \textit{Dataset} yang telah tersedia ini tetap digunakan, \textit{dataset} tersebut sudah terbagi dalam 5-\textit{fold}. Perubahan dilakukan untuk menyesuaikan format dan nama kolom  lebih sesuai.

\textit{Hyperparameter} yang digunakan ada dua, yaitu \textit{hyperparameter} pelatihan seperti \textit{learning rate}, dan \textit{hyperparameter} metode PEFT seperti \textit{rank} pada LoRA. Kedua \textit{hyperparameter} ini perlu ditentukan agar eksperimen konsisten. Untuk \textit{hyperaparemeter} pelatihan sudah ditentukan pada kakas IndoLEM. \textit{Hyperparameter} pelatihan ini digunakan kembali dengan \textit{hyperparameter} metode PEFT. Untuk \textit{hyperparameter} metode PEFT perlu dilakukan \textit{hyperparameter tuning} untuk menentukan \textit{hyperparameter} mana yang paling baik.

Terakhir, lingkungan pelatihan dibutuhkan untuk menjalankan proses pelatihan dan evaluasi. Lingkungan pelatihan membutuhkan GPU sehingga bisa memanfaatkan CUDA untuk mempercepat proses pelatihan. GPU yang khusus dibuat untuk pelatihan model merupakan pilihan yang paling tepat karena bisa menghemat waktu pelatihan. Terdapat banyak pilihan GPU untuk pelatihan berbasis \textit{cloud}, seperti Vast.AI dan Google Cloud Provider (GCP). Vast.AI mampu memberikan GPU sesuai permintaan setiap saaat, mempunyai pilihan GPU yang banyak, dan harga yang cukup terjangkau dibandingkan \textit{cloud provider} yang lain. Vast.AI menjadi pilihan yang tepat sebagai lingkungan pelatihan.


\section{Rancangan Solusi}
\label{sec:rancangan-solusi}

Dibangun sebuah rancangan solusi berupa skenario pengujian yang dilakukan dalam penelitian tugas akhir, yang dapat dilihat pada Gambar \ref{fig:rancangan-solusi}. Berdasarkan Gambar \ref{fig:rancangan-solusi}, tidak ditambahkan dataset baru, melainkan menggunakan dataset yang sudah disediakan pada IndoLEM. Teknik PEFT yang digunakan adalah LoRA (\textit{Low-Rank Adaptation}), \textit{Prefix-Tuning}, dan \textit{Bottleneck Adapter}. Hasil pengujian berupa kinerja dari teknik PEFT serta penggunaan sumber daya dari setiap eksperimen.

\begin{figure}[ht]
    \centering
    \includegraphics[height=0.5\textheight]{chapter-3/rancangan_solusi.png}
    \caption{Rancangan Solusi}
    \label{fig:rancangan-solusi}
\end{figure}

Sebelum bisa dilakukan pelatihan dan evaluasi model, kakas IndoLEM perlu dilakukan pengembangan ulang serta perlu dikembangkan metode PEFT pada kakas IndoLEM. Pengembangan ulang ini dilakukan agar pengembangan metode PEFT bisa dilakukan. Pengembangan ulang kakas IndoLEM akan menggunakan beberapa pustaka, salah satunya adalah Transformers dan Torch. Pustaka Transformers diperlukan untuk melakukan pemuatan model, praproses data, dan penggunaan Trainer untuk standardisasi. Lalu, pustaka Torch digunakan untuk menggunakan CUDA yang memanfaatkan GPU untuk melatih model. Selain itu, pustaka Wandb dan Huggingface juga akan dipakai untuk melakukan sinkronisasi pada \textit{cloud}. Dengan sinkronisasi ini, hasil pelatihan model dan hasil evaluasi dapat dilihat pada situsnya yang akan memudahkan proses eksperimen. 

Pengembangan metode PEFT pada kakas IndoLEM akan menggunakan pustaka Adapters. Model yang dimuat oleh pustaka Transformers perlu diinisiasi oleh pustaka Adapters untuk dapat berjalan menggunakan metode PEFT. Untuk dapat menjalankan proses pelatihan dengan metode PEFT, perlu menggunakan Trainer yang berasal dari pustaka Adapters yaitu AdapterTrainer yang akan melakukan \textit{freeze} pada parameter model dan akan menggunakan parameter yang sesuai dengan metode PEFT-nya tersebut.

Sesuai dengan banyaknya tugas evaluasi yang diimplementasikan, jumlah dari \textit{dataset} juga sebanyak tugas evaluasi tersebut, yaitu \nlptask. Setiap \textit{dataset} akan dilakukan 5-\textit{fold cross validation} yang membagi setiap dataset menjadi 5 bagian \textit{dataset} validasi yang berbeda. Metode \textit{cross validation} ini sudah dilakukan pada kakas IndoLEM, sehingga \textit{dataset} sudah terbagi menjadi 5-\textit{fold}. Untuk setiap \textit{fold}-nya, \textit{dataset} akan dilakukan praproses data, pelatihan model, dan evaluasi kinerja. Praproses data ini dilakukan dengan melakukan \textit{padding}, tokenisasi, dan juga mengatur \textit{mapping} dengan labelnya. 

Untuk proses pelatihan model akan dibagi menjadi dua, yaitu dengan menggunakan \textit{fine-tuning} tradisional dan menggunakan metode PEFT. Pelatihan model akan dilakukan pada lingkungan pelatihan yang sesuai yaitu pada lingkungkan \textit{cloud} yang menggunakan GPU khusus untuk pelatihan model. Pelatihan dengan \textit{fine-tuning} tradisional akan mengikuti \textit{hyperparameter} pada penelitian terkaitnya. Sedangkan, untuk metode PEFT akan dilakukan \textit{hyperparameter tuning} untuk metode PEFT-nya.

Selanjutnya, untuk setiap hasil pelatihan model akan dilakukan evaluasi pada dengan menggunakan \textit{dataset} validasi-nya. Metriks evaluasi yang akan digunakan tergantung pada tugas evaluasinya. Tugas evaluasi \textit{classification} yaitu NER dan \textit{sentiment analyisis} akan menggunakan \textit{accuracy} dan F1 \textit{score}. Sedangkan, tugas evaluasi \textit{generation} yaitu \textit{summarization} akan menggunakan ROGUE \textit{score}.



% \chapter{Implementasi, Eksperimen, dan Hasil Evaluasi Eksperimen}
Bab ini menjelaskan proses implementasi dari rancangan solusi yang telah dikaji pada Bab \ref{sec:chapter-3} dalam menyelesaikan permasalahan utama tugas akhir, yaitu pemanfaatan berbagai metode PEFT pada kakas evaluasi IndoLEM. Selain itu, dijelaskan juga lingkungan, implementasi, dan eksperimen.

Berdasarkan rancangan solusi yang diajukan pada subbab \ref{sec:rancangan-solusi}, pembangunan ulang kakas IndoLEM diperlukan untuk mengimplementasi setiap metode PEFT. Model yang dipilih yaitu IndoBERT untuk tugas evaluasi \textit{classification} dan IndoT5 untuk tugas evaluasi \textit{generation} dilatih pada lingkungan eksperimen. Model yang telah dilatih dengan metode \textit{fine-tuning} tradisional dan metode PEFT dievaluasi untuk membandingkan kinerja dari setiap metode pada setiap tugas evaluasinya.

\section{Lingkungan Eksperimen}
\label{sec:lingkungan-eksperimen}

Lingkungan eksperimen yang dipakai berbasis \textit{cloud} dengan menggunakan GPU khusus untuk pelatihan model, sehingga lingkungan ini khusus digunakan untuk melakukan pelatihan dan evaluasi model. Lingkungan ini dibagi menjadi dua bagian, yaitu untuk tugas evaluasi \textit{classification} dan \textit{generation}. Pembagian ini didasarkan pada kebutuhan memori yang berbeda antara kedua tugas tersebut. Tugas evaluasi \textit{generation} membutuhkan memori pada GPU yang lebih besar karena perlu memproses data dengan volume lebih banyak. Selain itu, tugas evaluasi \textit{generation} perlu melakukan generasi terhadap teks untuk menghasilkan evaluasi, sehingga membutuhkan memori yang lebih banyak.

\begin{table}[h]
    \vspace{0.25cm}
    \centering
    \caption{Lingkungan eksperimen}
    \label{table:lingkungan-eksperimen}
    \begin{tabular}{lcc}
        \toprule
        & \textbf{Tugas \textit{Classification}} & \textbf{Tugas \textit{Generation}} \\ 
        \midrule
        GPU          & Nvidia Tesla V100  & Nvidia A40       \\
        GPU RAM      & 16 GB              & 45 GB            \\
        TeraFLOPS    & 11.3               & 29.9             \\
        Lokasi       & Swedia             & Kroasia          \\
        Jumlah GPU   & 1                  & 1                \\
        \bottomrule
    \end{tabular}
\end{table}

Spesifikasi lengkap dari lingkungan eksperimen dapat dilihat pada tabel \ref{table:lingkungan-eksperimen}. Penggunaan dari lingkungan eksperimen tersebut adalah dengan mengakses GPU tersebut melalui SSH yang telah disediakan dari Vast.ai. \textit{Dataset} yang digunakan pada proses eksperimen akan dimuat pada GPU sehingga waktu pelatihan tidak terpengaruh oleh latensi yang disebabkan oleh lokasi GPU yang berbeda.

\begin{table}[h]
    \vspace{0.25cm}
    \centering
    \caption{Versi kakas dan perangkat lunak}
    \label{table:tools-version}
    \begin{tabular}{ll}
        \toprule
        \textbf{Kakas} & \textbf{Versi} \\ 
        \midrule
        Python                & 3.11        \\
        Transformers          & 4.40.2      \\
        Adapters              & 0.2.2       \\
        Datasets              & 2.20.0      \\
        Evaluate              & 0.4.2       \\
        Numpy                 & 2.0.0       \\
        Pandas                & 2.2.2       \\
        Seqeval               & 1.2.2       \\
        Torch                 & 2.3.1       \\
        Wandb                 & 0.17.3      \\
        nltk                  & 3.8.1       \\
        rouge\_score          & 0.1.2       \\
        huggingface\_hub      & 0.23.4      \\
        \bottomrule
    \end{tabular}
\end{table}

Versi dari kakas dan perangkat lunak yang digunakan dapat dilihat pada tabel \ref{table:tools-version}. Setiap eksperimen yang dilakukan menggunakan \textit{virtual environment} yang dibuat berdasarkan versi dari kakas dan perangkat lunak tersebut untuk memastikan konsistensi pada eksperimen. Lingkungan dibuat dengan menggunakan \textit{virtual environment} dari Conda.

\section{Pengembangan Ulang Kakas IndoLEM}
\label{sec:pengembangan-kakas}

Pengembangan ulang kakas IndoLEM ini diperlukan untuk memperbarui dan memperbaiki infrastruktur dari kakas tersebut. Pengembangan ini dilakukan pada lingkungan lokal pribadi, untuk keperluan eksperimen dilakukan pada lingkungan \textit{cloud} seperti yang sudah disebutkan pada subbab \ref{sec:lingkungan-eksperimen}.

\subsection{Persiapan Lingkungan Pengembangan}

Implementasi diawali dengan menyiapkan lingkungan pengembangan, untuk menyiapkan lingkungan pengembangan dilakukan dengan melakukan instalasi terhadap perangkat lunak dan pustaka yang dibutuhkan. Berikut merupakan \textit{requirements.txt} yang berisi pustaka yang perlu dilakukan intalasi yang bisa dilihat pada tabel \ref{table:requirements}

\begin{table}
    \caption{Tabel \textit{requirements.txt}}
    \label{table:requirements}
    \begin{lstlisting}[language=bash]
    accelerate
    adapters==0.2.2
    datasets
    evaluate
    huggingface-hub
    numpy
    pandas
    scikit-learn
    seqeval
    torch
    transformers==4.40.*
    wandb
    nltk
    rouge_score
    indobenchmark-toolkit
    \end{lstlisting}
\end{table}

Untuk memudahkan pengembangan perlu membuat \textit{virtual environment} untuk mengenkapsulasi pustaka yang dipakai. \textit{Virtual environment} bisa dibuat dengan menggunakan Conda atau pustaka virtualenv. Untuk tugas akhir ini, digunakan Conda. \textit{Command} yang digunakan dapat dilihat pada tabel \ref{table:command-virtualenv}.

\begin{table}
    \caption{Tabel \textit{command virtual environment}}
    \label{table:command-virtualenv}
    \begin{lstlisting}[language=bash]
    # Membuat virtual environment
    conda create env -n indolem python=3.11
    # Menjalankan virtual environment
    conda activate indolem
    # Melakukan instalasi pustaka 
    pip install -r requirements.txt
    \end{lstlisting}
\end{table}

\textit{Command} tersebut akan membuat \textit{virtual environment}, mengakitfkannya, dan melakukan instalasi pada versi Python dan pustaka yang dibutuhkan.

\subsection{Penguraian Argumen}
\label{sec:parse-arg}

Penguraian argumen dilakukan dengan menggunakan modul HfArgumentParser dari pustaka Transformers. Modul ini digunakan untuk memudahkan penguraian argumen karena argumen yang dipakai untuk model, pelatihan, dan evaluasi sudah diimplementasikan dari modul tersebut. Untuk argumen tambahan yang spesifik pada suatu tugas evaluasi perlu ditambahkan secara manual, sebagai contoh pada tugas evaluasi NER terdapat argumen \textit{return\_entity\_level\_metrics} yaitu menggunakan metriks evaluasi untuk level entitasnya.

\begin{table}
    \caption{Tabel kode HfArgumentParser}
    \label{table:hfargumentparser}
    \begin{lstlisting}[language=python]
    parser = HfArgumentParser(
        [
            ModelArguments,
            DataTrainingArguments,
            TrainingArguments,
            AdapterArguments,
            WandbArguments,
        ]
    )

    if len(sys.argv) == 2 and sys.argv[1].endswith(".json"):
        # If we pass only one argument to the script and it's the path to a json file,
        # let's parse it to get our arguments.
        model_args, data_args, training_args, adapter_args, wandb_args = (
            parser.parse_json_file(json_file=os.path.abspath(sys.argv[1]))
        )
    else:
        model_args, data_args, training_args, adapter_args, wandb_args = (
            parser.parse_args_into_dataclasses()
        )
    \end{lstlisting}
\end{table}

Berdasarkan tabel \ref{table:hfargumentparser} terdapat 5 jenis argumen, yaitu model, \textit{data training}, pelatihan, \textit{adapter}, dan wandb. Argumen model digunakan untuk pemuatan model. Argumen \textit{data training} digunakan untuk pemuatan \textit{file} (latih, evaluasi, dan uji), \textit{padding}, maksimum sampel, maksimum sekuens, dan lain sebagainya yang berhubungan dengan \textit{dataset}. Lalu, argumen pelatihan merupakan \textit{hyperparemeter} yang digunakan pada proses pelatihan seperti \textit{learning rate, epoch, batch size}, dan lain sebagainya. Selanjutnya, argumen \textit{adapter} berfungsi sebagai konfigurasi untuk jenis metode PEFT beserta \textit{hyperparameter} yang digunakannya. Terakhir, merupakan argumen wandb yang digunakan untuk konfigurasi sinkronisasi pada hasil evaluasi ke \textit{cloud}.

\subsection{Praproses \textit{Dataset}}
\label{sec:praproses}

Praproses \textit{dataset} diperlukan agar model dapat memproses \textit{dataset} dengan konteks yang sesuai. Konteks yang dimaksud adalah, model dapat memahami \textit{metadata} dari \textit{dataset}, sehingga mampu mengetahui setiap bagian data yang merupakan bagian dari teks atau label. Praproses \textit{dataset} ini berbeda untuk setiap tugas evaluasi karena karakteristik dari tugas evaluasinya yang berbeda juga.

\subsubsection{Praproses Data NER}

Tugas evaluasi NER merupakan jenis tugas \textit{classification} pada level token yang berarti setiap token akan diklasifikasikan dengan label tertentu. Pada kakas IndoLEM, untuk tugas evaluasi NER terdapat dua jenis \textit{dataset}, yaitu NERUI dan NERUGM. Kedua \textit{dataset} tersebut mempunyai label yang berbeda berikut merupakan label pada setiap \textit{dataset}-nya dapat dilihat pada tabel \ref{table:label-ner}.

\begin{table}[h]
    \vspace{0.25cm}
    \caption{Tabel label \textit{dataset} NER}
    \label{table:label-ner}
    \begin{center}
        \begin{tabular}{|l|l|}
            \hline \rowcolor{black!10}
            \multicolumn{1}{|c|}{\textbf{NERUI}} & \multicolumn{1}{|c|}{\textbf{NERUI}} \\ \hline
            B-LOCATION & B-LOCATION \\ \hline
            B-ORGANIZATION & B-ORGANIZATION \\ \hline
            B-PERSON & B-PERSON \\ \hline
            - & B-QUANTITY \\ \hline
            - & B-TIME \\ \hline
            I-LOCATION & I-LOCATION \\ \hline
            I-ORGANIZATION & I-ORGANIZATION \\ \hline
            I-PERSON & I-PERSON \\ \hline
            - & I-QUANTITY \\ \hline
            - & I-TIME \\ \hline
            O & O \\ \hline
        \end{tabular}
    \end{center}
\end{table}

Label tersebut diperlukan untuk melakukan \textit{mapping} antara id dengan labelnya yang akan digunakan pada konfigurasi model. \textit{Padding} juga dilakukan berdasarkan panjang sekuens maksimum yang didefinisikan melaluai argumen \textit{data training}. Selanjutnya, tokenisasi akan dilakukan terhadap \textit{dataset} tersebut. Implementasi dari fungsi tokenisasi tersebut dapat dilihat pada tabel \ref{table:tokenisasi-ner}.

\begin{table}
    \caption{Tabel fungsi tokenisasi NER}
    \label{table:tokenisasi-ner}
    \begin{lstlisting}[language=python]
    # Tokenize all texts and align the labels with them.
    def tokenize_and_align_labels(examples):
        tokenized_inputs = tokenizer(
            examples[text_column_name],
            padding=padding,
            truncation=True,
            max_length=data_args.max_seq_length,
            # We use this argument because the texts in our dataset are lists of words (with a label for each word).
            is_split_into_words=True,
        )
        labels = []
        for i, label in enumerate(examples[label_column_name]):
            word_ids = tokenized_inputs.word_ids(batch_index=i)
            previous_word_idx = None
            label_ids = []
            for word_idx in word_ids:
                # Special tokens have a word id that is None. We set the label to -100 so they are automatically
                # ignored in the loss function.
                if word_idx is None:
                    label_ids.append(-100)
                # We set the label for the first token of each word.
                elif word_idx != previous_word_idx:
                    label_ids.append(label_to_id[label[word_idx]])
                # For the other tokens in a word, we set the label to either the current label or -100, depending on
                # the label_all_tokens flag.
                else:
                    if data_args.label_all_tokens:
                        label_ids.append(b_to_i_label[label_to_id[label[word_idx]]])
                    else:
                        label_ids.append(-100)
                previous_word_idx = word_idx

            labels.append(label_ids)
        tokenized_inputs["labels"] = labels
        return tokenized_inputs
    \end{lstlisting}
\end{table}

Tokenisasi dilakukan dengan menggunakan modul Tokenizer pada pustaka Transformers. Fungsi tokenisasi pada tabel \ref{table:tokenisasi-ner} mengabaikan token spesial [CLS] dan [SEP] yang muncul dari hasil tokenisasi model berbasi BERT. Token spesial tersebut diubah menjadi -100 agar bisa diabaikan dari perhitungan metriks evaluasinya. Sehingga, hanya token yang berkorespondensi terhadap sebuah label saja yang dapat dievaluasi.

\subsubsection{Praproses Data \textit{Sentiment Analysis}}

Tugas evaluasi \textit{sentiment analysis} merupakan jenis tugas \textit{classification} sama seperti NER, hanya perbedaannya pada level klasifikasi yang digunakan. Jika pada NER klasifikasi dilakukan pada level token, pada \textit{sentiment analysis} klasifikasi dilakukan pada level teks yang merupakan gabungan dari beberapa token. \textit{Dataset} dari \textit{sentiment analysis} berupa \textit{file} CSV yang berisi dari dua kolom yaitu \textit{sentence} dan \textit{labels}. Kolom \textit{sentence} merupakan teks yang mengandung sentiment tertentu. Sementara kolom \textit{labels} menandakan sentimen dari teks tersebut, dengan nilai 0 berupa sentimen negatif dan  nilai 1 berupa sentiment positif.

\begin{table}
    \caption{Tabel fungsi tokenisasi \textit{sentiment analysis}}
    \label{table:tokenisasi-sentiment}
    \begin{lstlisting}[language=python]
        def preprocess_function(examples):
            # Tokenize the texts
            args = (
                (examples[sentence1_key],)
                if sentence2_key is None
                else (examples[sentence1_key], examples[sentence2_key])
            )
            result = tokenizer(
                *args, padding=padding, max_length=max_seq_length, truncation=True
            )

            # Map labels to IDs
            if label_to_id is not None and "labels" in examples:
                result["labels"] = [
                    (label_to_id[l] if l != -1 else -1) for l in examples["labels"]
                ]
            return result
    \end{lstlisting}
\end{table}

Pada tugas evaluasi \textit{sentiment analysis} juga diperlukan \textit{mapping} antara label dengan idnya, sehingga model mampu mengetahui label yang dilatih. Fungsi tokenisasi \textit{sentiment analysis} yang dapat dilihat pada tabel \ref{table:tokenisasi-sentiment} menggunakan modul Tokenizer dari pustaka Transformers. Fungsi tersebut terdapat \textit{sentence1\_key} dan juga \textit{sentence2\_key}, yang berguna apabila terdapat dua kolom teks pada \textit{dataset}, tetapi pada IndoLEM \textit{dataset sentiment analysis} hanya mempunyai satu kolom teks. Sehingga, \textit{sentence2\_key} tidak digunakan. 

\subsubsection{Praproses Data \textit{Summarization}}

Tugas evaluasi \textit{summarization} merupakan jenis tugas \textit{generation} yang berbeda dengan jenis tugas \textit{classification}. Tugas \textit{generation} memerlukan generasi teks untuk dapat dilakukan evaluasi, pada konteks ini yaitu membuat ringkasan dari teks. \textit{Dataset} yang digunakan pada IndoLEM untuk tugas evaluasi \textit{summarization} adalah \textit{dataset} IndoSUM. \textit{Dataset} ini berisi \textit{file} dengan format JSONL dengan masing-masing objek dapat dilihat pada tabel \ref{table:data-summarization}.

\begin{table}
    \vspace{0.25cm}
    \caption{Tabel data \textit{summarization}}
    \label{table:data-summarization}
    \begin{center}
        \begin{tabular}{|l|l|}
            \hline \rowcolor{black!10}
            \multicolumn{1}{|c|}{\textbf{Objek}} & \multicolumn{1}{|c|}{\textbf{Deskripsi}} \\ \hline
            id & nilai unik untuk setiap artikel \\ \hline
            paragraphs & \textit{list} dari paragraf yang berasal dari artikel \\ \hline
            summary & ringkasan dari artikel, disimpan sebagai \textit{list} dari kalimat \\ \hline 
            gold\_labels & label ekstraktif dari setiap kalimat pada artikel \\ \hline
            category & kategori dari artikel \\ \hline
            source & sumber artikel \\ \hline
            source\_url & URL dari artikel \\ \hline
        \end{tabular}
    \end{center}
\end{table}

Dari data yang dapat dilihat pada tabel \ref{table:data-summarization}, terdapat 7 objek  pada \textit{dataset}. Namun, pada tugas evaluasi ini hanya akan digunakan dua objek yaitu \textit{paragraphs} dan \textit{summary} karena tugas \textit{summarization} membutuhkan teks asli sebelum diringkas dan hasil ringkasannya. Objek yang lain selain dua yang digunakan bisa diabaikan saja. \textit{Dataset} ini perlu dipraproses terlebih dahulu menggunakan fungsi yang dapat dilihat pada tabel \ref{table:tokensisasi-summ}.

\begin{table}
    \caption{Tabel fungsi tokenisasi \textit{summarization}}
    \label{table:tokensisasi-summ}
    \begin{lstlisting}[language=python]
    def preprocess_function(examples):
        # Preprocess data for indosum
        if is_indosum:
            examples[text_column] = paragraph_to_text(examples[text_column])
            examples[summary_column] = summary_to_text(examples[summary_column])

        inputs, targets = [], []
        for i in range(len(examples[text_column])):
            if examples[text_column][i] and examples[summary_column][i]:
                inputs.append(examples[text_column][i])
                targets.append(examples[summary_column][i])

        inputs = [prefix + inp for inp in inputs]
        model_inputs = tokenizer(
            inputs,
            max_length=data_args.max_source_length,
            padding=padding,
            truncation=True,
        )

        # Tokenize targets with the `text_target` keyword argument
        labels = tokenizer(
            text_target=targets,
            max_length=max_target_length,
            padding=padding,
            truncation=True,
        )

        # If we are padding here, replace all tokenizer.pad_token_id in the labels by -100 when we want to ignore
        # padding in the loss.
        if padding == "max_length" and data_args.ignore_pad_token_for_loss:
            labels["input_ids"] = [
                [(l if l != tokenizer.pad_token_id else -100) for l in label]
                for label in labels["input_ids"]
            ]

        model_inputs["labels"] = labels["input_ids"]
        return model_inputs
    \end{lstlisting}
\end{table}

Berdasarkan tabel \ref{table:tokensisasi-summ}, tokenisasi dilakukan dengan mengolah objek \textit{paragraphs} menjadi \textit{list} dari teks. Setiap paragraf terdiri dari beberapa kalimat, dan setiap kalimat terdiri dari beberapa teks. Untuk \textit{text\_column} yang merupakan objek \textit{paragraphs} akan diolah menjadi \textit{list} dari teks. Hal yang sama berlaku juga untuk \textit{summary\_column}, hanya saja objek ini merupakan \textit{summary} yang berisi kalimat. Selain itu, tugas evaluasi \textit{summarization} ini terdapat \textit{prefix} yang berguna pada beberapa model tertentu. Tokenisasi juga menggunakan modul \textit{tokenizer} dari pustaka Transformers.

\subsection{Pemuatan Konfigurasi Model}

Untuk dapat melakukan pelatihan dan evaluasi terhadap model, terlebih dahulu model perlu dimuat. Pemuatan model ini menggunakan konfigurasi yang dimuat dari penguraian argumen pada subbab \ref{sec:parse-arg}. Argumen model akan dimuat ke dalam model sehingga model bisa dipakai. Model akan dimuat dengan menggunakan modul yang relevan dari pustaka Transformers.

\begin{table}
    \caption{Tabel kode pemuatan model \textit{summarization}}
    \label{table:kode-model-summ}
    \begin{lstlisting}[language=python]
    model = AutoModelForSeq2SeqLM.from_pretrained(
        model_args.model_name_or_path,
        from_tf=bool(".ckpt" in model_args.model_name_or_path),
        config=config,
        cache_dir=model_args.cache_dir,
        revision=model_args.model_revision,
        token=True if model_args.token else None,
    )
    \end{lstlisting}
\end{table}

Pada tabel \ref{table:kode-model-summ} dimuat model AutoModelForSeq2SeqLM yang merupakan model \textit{sequence to sequence} yang merupakan model \textit{encoder decoder}. Modul AutoModel ini akan menghasilkan model berdasarkan dari konfigurasi \textit{model\_args} yang dimasukkan pada modul tersebut. Untuk tugas evaluasi \textit{summarization} yang menggunakan model IndoT5 akan menghasilkan model tersebut juga. Berbeda untuk tugas \textit{classification}, cukup dengan menyatakan AutoModel saja sudah cukup untuk menghasilkan model yang sesuai. Pada konteks ini, model yang digunakan untuk NER dan \textit{sentiment analysis} adalah IndoBERT yang berbasis BERT.

\subsection{Perhitungan Metriks Evaluasi}
\label{sec:metriks-evaluasi}

Metriks evaluasi yang digunakan pada setiap tugas evaluasi tidak sama semuanya. Metriks yang digunakan untuk tugas \textit{classification} adalah \textit{accuracy} dan F1 \textit{score}. Sedangkan, pada tugas \textit{generation} adalah ROUGE \textit{score}. Perhitungan metriks evaluasi ini dilakukan saat proses evaluasi dengan menggunakan fungsi \textit{compute\_metrics}.

\begin{table}
    \caption{Tabel fungsi \textit{compute\_metrics}}
    \label{table:compute-metrics}
    \begin{lstlisting}[language=python]
    def compute_metrics(p):
        logits, labels = p
        predictions = np.argmax(logits, axis=1)

        return {
            "accuracy": accuracy_score(labels, predictions),
            "precision": precision_score(labels, predictions, average="macro"),
            "recall": recall_score(labels, predictions, average="macro"),
            "f1": f1_score(labels, predictions, average="macro"),
        }
    \end{lstlisting}
\end{table}

Pada tabel \ref{table:compute-metrics}, fungsi perhitungan metriks evaluasinya merupakan fungsi yang dipakai untuk tugas \textit{classification}. Fungsi tersebut memanfaatkan modul seqeval untuk melakukan perhitungan setiap metriksnya. Terdapat perbeedaan pada fungsi yang dipakai pada tugas \textit{classifcation}, NER menggunakan \textit{entity\_level\_metrics} yang berarti perhitungan evaluasinya berdasarkan pada entitasnya yang berbeda dengan \textit{sentiment analysis}. Untuk tugas \textit{generation} yaitu \textit{summarization} menggunakan ROUGE \textit{score}.

\subsection{Pemuatan Trainer untuk Pelatihan dan Evaluasi Model}

Untuk pelatihan dan evaluasi model dibutuhkan standardisasi agar kode yang digunakan pada setiap tugas tidak berbeda. Modul Trainer dari pustaka Transformers digunakan untuk pelatihan dan evaluasi model. Setiap komponen yang sudah disebutkan sebelumnya akan dimuat pada modul Trainer. Setiap tugas evaluasi akan menggunakan implementasi yang sama yang bisa dilihat pada tabel \ref{table:kode-trainer}.

\begin{table}
    \caption{Tabel kode implementasi Trainer}
    \label{table:kode-trainer}
    \begin{lstlisting}[language=python]
    trainer = Trainer(
        model=model,
        args=training_args,
        train_dataset=train_dataset if training_args.do_train else None,
        eval_dataset=eval_dataset if training_args.do_eval else None,
        tokenizer=tokenizer,
        data_collator=data_collator,
        compute_metrics=compute_metrics,
    )

    trainer.train(resume_from_checkpoint=checkpoint)
    trainer.evaluate()
    \end{lstlisting}
\end{table}

Kode implementasi Trainer pada tabel \ref{table:kode-trainer} memuat modul Trainer dan memanggil metode \textit{train} dan \textit{evaluate} untuk melakukan pelatihan dan evaluasi. Modul Trainer membutuhkan beberapa argumen yang dimuat. Argumen tersebut dimuat dari komponen-komponen yang sudah disebutkan pada subbab sebelumnya. Argumen untuk pelatihan, dan model didapat dari hasil penguraian argumen pada \ref{sec:parse-arg}. Untuk \textit{tokenizer} dan \textit{dataset} pelatihan dan evaluasi dimuat dari hasil praproses data pada \ref{sec:praproses}. Lalu, untuk \textit{compute\_metrics} fungsinya dimuat dari metriks evaluasi pada \ref{sec:metriks-evaluasi}. 

\section{Pengembangan Metode PEFT pada Kakas IndoLEM}

Pengembangan metode PEFT pada kakas IndoLEM menggunakan pustaka Adapters. Pustaka tersebut tidak hanya mengimplementasikan setiap metode saja, tetapi terdapat fungsionalitas untuk menggabungkan beberapa metode PEFT. Pada subbab sebelumnya yaitu subbab \ref{sec:pengembangan-kakas}, kakas IndoLEM dikembangkan ulang agar mampu untuk mengimplementasikan metode PEFT. Implementasi tersebut akan mengubah beberapa komponen, yaitu komponen model dan trainer. Metode PEFT akan melakukan \textit{freeze} pada parameter model sehingga pada proses pelatihan tidak akan mengubah parameter model melainkan parameter dari metode PEFT tersebut.

\begin{table}[h]
    \caption{Tabel kode pemuatan model Adapters}
    \label{table:kode-model-adapters}
    \begin{lstlisting}[language=python]
    model = AutoAdapterModel.from_pretrained(
        model_args.model_name_or_path,
        from_tf=bool(".ckpt" in model_args.model_name_or_path),
        config=config,
        cache_dir=model_args.cache_dir,
        revision=model_args.model_revision,
        token=True if model_args.token else None,
        ignore_mismatched_sizes=model_args.ignore_mismatched_sizes,
    )
    \end{lstlisting}
\end{table}

Pada kode pemuatan model Adapters yang bisa dilihat pada tabel \ref{table:kode-model-adapters}, model dimuat dengan modul AutoAdapterModel dari pustaka Adapters. Hal ini dilakukan untuk \textit{support} metode PEFT yang lebih baik berdasarkan dari dokumentasi dari pustakanya. Berbeda dari tabel \ref{table:kode-model-summ} yang menggunakan AutoModel dari pustaka Transformers, AutoAdapterModel memberikan fungsionalitas yang lebih baik untuk metode PEFT, salah satunya adalah untuk membandingkan parameter yang akan dilatih dibanding dengan parameter asli dari modelnya.

\begin{table}[h]
    \caption{Tabel kode implementasi AdapterTrainer}
    \label{table:kode-adaptertrainer}
    \begin{lstlisting}[language=python]
    setup_adapter_training(model, adapter_args)
    trainer = AdapterTrainer(
        model=model,
        args=training_args,
        train_dataset=train_dataset if training_args.do_train else None,
        eval_dataset=eval_dataset if training_args.do_eval else None,
        tokenizer=tokenizer,
        data_collator=data_collator,
        compute_metrics=compute_metrics,
    )

    trainer.train(resume_from_checkpoint=checkpoint)
    trainer.evaluate()
    \end{lstlisting}
\end{table}

Selanjutnya, untuk proses pelatihan perlu disiapkan untuk menggunakan metode PEFT. Implementasinya menggunakan modul AdapterTrainer dari pustaka Adapters juga. Model perlu disiapkan untuk pelatihan dengan metode PEFT dengan menggunakan fungsi \textit{setup\_adapter\_training}. Fungsi tersebut akan melakukan \textit{freeze} pada parameter model dan hanya akan mengubah parameter dari metode PEFT yang digunakan. Argumen untuk \textit{adapter} dimuat dari hasil pengurain argumen pada \ref{sec:parse-arg}, argumen tersebut berisi jenis metode PEFT serta \textit{hyperparameter} yang digunakan pada proses pelatihan. Proses pelatihan metode PEFT berbeda dengan pelatihan dengan metode \textit{fine-tuning} tergantung dari karakteristik dari metode tersebut. Pustaka Adapters mengimplementasikannya dengan menambahkan semacam \textit{adapter} pada model, sehingga \textit{adapter} tersebut yang dilatih bukan modelnya. Parameter dari \textit{adapter} tersebut relatif lebih kecil daripada parameter model.

\section{Pelatihan Model}
\label{sec:pelatihan-model}

Pelatihan model dilakukan pada lingkungan eksperimen yang telah disebutkan pada subbab \ref{sec:lingkungan-eksperimen}. Pelatihan dilakukan dengan \textit{bash script} yang  dijalankan pada \textit{shell} untuk memudahkan reka ulang dari eksperimen tersebut. Untuk setiap tugas evaluasi terdapat \textit{bash script} yang  menjalankan semua eksperimen, yaitu \textit{fine-tuning}, \methodPEFT.

\begin{figure}[h]
    \centering
    \caption{Struktur \textit{file} eksperimen}
    \label{fig:file-eksperimen}
    \begin{forest}
        for tree={
            font=\ttfamily,
            grow'=0,
            child anchor=west,
            parent anchor=south,
            anchor=west,
            calign=first,
            edge path={
                \noexpand\path [draw, \forestoption{edge}]
                (!u.south west) +(7.5pt,0) |- node[fill,inner sep=1.25pt] {} (.child anchor)\forestoption{edge label};
            },
            before typesetting nodes={
                if n=1
                    {insert before={[,phantom]}}
                    {}
            },
            fit=band,
            before computing xy={l=15pt},
        }
    [run\_sentiment.sh
        [script
            [run\_sentiment\_base.sh]
            [run\_sentiment\_lora.sh]
            [run\_sentiment\_pt.sh]
            [run\_sentiment\_seq\_bn.sh]
            [run\_sentiment\_unipelt.sh]
        ]
    ]
    \end{forest}
\end{figure}

Pada gambar \ref{fig:file-eksperimen}, dengan melakukan pemanggilan pada {\ttfamily run\_sentiment.sh}  menjalankan semua eksperimen yang ada pada folder {\ttfamily script}. Setiap \textit{script}  menjalankan eksperimen sesuai dari konfigurasi yang telah ditentukan pada \textit{script} tersebut. Konfigurasi yang ditentukan tersebut  diterima sebagai argumen pada kakas IndoLEM. Untuk setiap \textit{hyperparameter} yang digunakan untuk pelatihan dapat dilihat pada lampiran \ref{appendix:hyperparameter-train}.

Untuk pelatihan model NER terdapat dua \textit{dataset} yang  digunakan yaitu NERUGM dan NERUI. Sedangkan, untuk tugas \textit{sentiment analysis} dan \textit{summarization} menggunakan satu dataset. Selain itu, untuk tugas NER terdapat argumen "return entity level metrics" dengan nilai "true" yang berarti penilaian metriks evaluasinya  dinilai pada level entitas. Ini mengikuti eksperimen yang sebelumnya sudah dilakukan pada IndoLEM.

Terdapat perbedaan pada \textit{hyperparameter} yang digunakan untuk tugas \textit{summarization} karena argumen yang digunakan mengikuti eksperimen pada IndoT5 bukan pada IndoLEM. Terdapat juga argumen "predict with generate" dengan nilai "true" karena tugas \textit{summarization} perlu menghasilkan hasil ringkasan untuk dapat dievaluasi. Selain itu, nilai argumen "bf16" bernilai "true" karena presisi "bf16" dianggap bisa mempercepat proses pelatihan untuk model T5 yang digunakan. Pada tugas ini perlu menambahkan argumen "source prefix" yang bernilai "summarize: " khusus untuk penggunaan model T5 pada tugas \textit{summarization}.

\begin{table}[h]
    \centering
    \caption{\textit{Hyperparameter} metode PEFT}
    \label{table:hyperparameter-PEFT}
    \begin{tabular}{l|l|c}
        \toprule
        \textbf{Metode} & \textbf{Argumen} & \textbf{Nilai} \\
        \midrule
        LoRA & \texttt{rank} & [8, 16] \\
        Prefix Tuning & \texttt{prefix\_length} & [5, 50] \\
        Adapter & \texttt{reduction\_factor} & [64, 16] \\
        UniPELT & \multicolumn{1}{c|}{-}  & - \\
        \bottomrule
    \end{tabular}
\end{table}

Pada pelatihan dengan metode PEFT  menggunakan \textit{hyperparameter} pelatihan seperti pada lampiran \ref{appendix:hyperparameter-train}. Untuk setiap metode PEFT terdapat \textit{hyperparameter} yang sesuai dengan metode PEFT tersebut. Perlu dilakukan \textit{hyperparameter tuning} untuk \textit{hyperparameter} metode PEFT. Seperti yang bisa dilihat pada tabel \ref{table:hyperparameter-PEFT}, untuk metode LoRA dilakukan eksperimen dengan argumen \texttt{ranks} dengan nilai 8 dan 16. Lalu, untuk metode Prefix Tuning dilakukan eksperimen dengan argumen \texttt{prefix\_tuning} dengan nilai 5 dan 50. Adapter akan menggunakan argumen \texttt{reduction\_factor} dengan nilai 64 dan 16. Selain itu, UniPELT tidak menggunakan argumen tambahan, hal ini dilakukan karena metode tersebut memang spesifik diimplementasikan pada pustaka Adapters sesuai dengan penelitian terkaitnya.

\input{chapters/chapter-4/05-ner.tex}
\section{Evaluasi Tugas \textit{Sentiment Analysis}}

\section{Evaluasi Tugas \textit{Summarization}}


% \chapter{Penutup}

Bab Kesimpulan dan Saran akan menjadi bagian akhir dan penutup dari penelitian tugas akhir ini. Bab ini akan membahas kesimpulan yang berisi ketercapaian tujuan penelitian tugas akhir dengan permasalahan yang diselesaikan dalam penelitian tugas akhir. Selain itu, bab ini akan membahas saran yang dapat dilakukan untuk pengembangan atau penelitian selanjutnya.

\section{Kesimpulan}

\section{Saran}

%---------------------------------------------------------------%

% Daftar pustaka
\printbibliography

% Setting judul lampiran
\titlespacing*{\chapter}{0pt}{0pt}{0pt}
\titlespacing*{\section}{0pt}{0pt}{*1}

% Setting judul anak lampiran
\titleformat*{\section}{\bfseries}

\appendix

\chapter{\textit{Hyperparameter} pelatihan}

\begin{table}[h]
    \centering
    \caption{\textit{Hyperparameter} pelatihan}
    \label{appendix:hyperparameter-train}
    \resizebox{\textwidth}{!}{
        \begin{tabular}{|c|l|c|}
            \hline 
            \textbf{Tugas} & \multicolumn{1}{|c|}{\textbf{Argumen}} & \textbf{Nilai} \\ \hline
            \multirow{9}{*}{\textit{NER}} & \texttt{model} & "indolem/indobert-base-uncased" \\ \cline{2-3}
                                          & \texttt{train\_batch\_size} & 16 \\ \cline{2-3}
                                          & \texttt{eval\_batch\_size} & 64 \\ \cline{2-3}
                                          & \texttt{epochs} & 100 \\ \cline{2-3}
                                          & \texttt{learning\_rate} & 5e-5 \\ \cline{2-3}
                                          & \texttt{max\_sequence\_length} & 128 \\ \cline{2-3}
                                          & \texttt{text\_column\_names} & "tokens" \\ \cline{2-3}
                                          & \texttt{label\_column\_names} & "ner\_tags" \\ \cline{2-3}
                                          & \texttt{return\_entity\_level\_metrics} & true \\ \cline{2-3}
                                          & \texttt{seed} & 42 \\ \hline
            \multirow{7}{*}{\textit{Sentiment Analysis}} & \texttt{model} & "indolem/indobert-base-uncased" \\ \cline{2-3}
                                                         & \texttt{batch\_size} & 30 \\ \cline{2-3}
                                                         & \texttt{epochs} & 20 \\ \cline{2-3}
                                                         & \texttt{learning\_rate} & 5e-5 \\ \cline{2-3}
                                                         & \texttt{max\_sequence\_length} & 200 \\ \cline{2-3}
                                                         & \texttt{seed} & 42 \\ \cline{2-3}
                                                         & \texttt{label\_names} & "labels" \\ \hline
            \multirow{14}{*}{\textit{Summarization}} & \texttt{model} & "lazarusnlp/indonanot5-base" \\ \cline{2-3}
                                                    & \texttt{train\_batch\_size} & 4 \\ \cline{2-3}
                                                    & \texttt{eval\_batch\_size} & 8 \\ \cline{2-3}
                                                    & \texttt{epochs} & 5 \\ \cline{2-3}
                                                    & \texttt{learning\_rate} & 1e-5 \\ \cline{2-3}
                                                    & \texttt{max\_source\_length} & 512 \\ \cline{2-3}
                                                    & \texttt{max\_target\_length} & 512 \\ \cline{2-3}
                                                    & \texttt{num\_beams} & 5 \\ \cline{2-3}
                                                    & \texttt{weight\_decay} & 0.01 \\ \cline{2-3}
                                                    & \texttt{patience} & 5 \\ \cline{2-3}
                                                    & \texttt{seed} & 42 \\ \cline{2-3}
                                                    & \texttt{text\_column} & paragraphs \\ \cline{2-3}
                                                    & \texttt{summary\_column} & summary \\ \cline{2-3}
                                                    & \texttt{source\_prefix} & "summarize: " \\ \cline{2-3}
                                                    & \texttt{pad\_to\_max\_length} & true \\ \cline{2-3}
                                                    & \texttt{predict\_with\_generate} & true \\ \cline{2-3}
                                                    & \texttt{bf16} & true \\ \hline
        \end{tabular}
    }
\end{table}


\end{document}
