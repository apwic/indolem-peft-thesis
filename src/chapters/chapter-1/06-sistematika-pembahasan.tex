\section{Sistematika Pembahasan}

Konten dari Tugas Akhir ini  dibagi menjadi lima bab sebagai berikut.
\begin{enumerate}
    \item Pendahuluan

        Pada Bab I dijelaskan gagasan utama dari tugas akhir ini yang berisi latar belakang terkait IndoLEM dan keterbatasannya pada metode yang digunakan yaitu, sehingga terdapat peluang untuk digunakan metode lain. Lalu, dijelaskan juga mengenai rumusan masalah dan tujuan yang perlu dicapai pada tugas akhir ini. Selain itu, terdapat batasan masalah terkait tugas evaluasi, metode PEFT, \textit{dataset}, dan kakas yang digunakan.

    \item Studi Literatur

        Selanjutnya, Bab II  menjelaskan hasil studi literatur yang berkaitan dengan pemanfaatan metode PEFT pada tugas evaluasi IndoLEM. Studi literatur berisi penjelasan mengenai tugas NLP, arsitektur Transformer, \textit{pre-trained model}, PEFT, metrik evaluasi, penelitian terkait, dan kakas pengembangan.

    \item Analisis Persoalan dan Desain Eksperimen

        Pada Bab III  dijelaskan analisis persoalan yang dijawab dengan analisis solusi. Selanjutnya, desain eksperimen dijelaskan terkait eksperimen yang dilakukan pada tugas akhir ini. Eksperimen dilakukan dengan menguji 3 tugas evaluasi, yaitu \nlptask dengan 5 metode. Metode yang digunakan adalah \textit{fine-tuning}, \methodPEFT. Selain itu, dilakukan variasi terhadap metode PEFT, sehingga total metode beserta variasinya adalah 8 metode.

    \item Implementasi, Eksperimen, dan Hasil Evaluasi Eksperimen

        Bab IV ini berisi kajian terhadap implementasi yang telah dibuat. Bagian implementasi dijelaskan terkait pengembangan \textit{script} dan pemanfaatan metode PEFT. Bagian eksperimen dijelaskan terkait lingkungan eksperimen, pelatihan model, dan evaluasi model. Evaluasi hasil eksperimen dijelaskan terkait dengan waktu pelatihan, parameter model, dan metrik yang digunakan.

    \item Kesimpulan dan Saran

        Bab V menutup tugas akhir ini yang berisi dengan kesimpulan dan saran. Kesimpulan pada bab ini menjawab rumusan masalah. Selain itu, saran yang bisa dilakukan untuk penelitian selanjutnya dijelaskan pada bab ini.
\end{enumerate}
