\section{Sistematika Pembahasan}

Konten dari Tugas Akhir ini  dibagi menjadi lima bab sebagai berikut.
\begin{enumerate}
    \item Pendahuluan

    Pada Bab I  dijelaskan gagasan utama dari tugas akhir ini yang berisi dari latar belakang, rumusan masalah, tujuan, batasan, metodologi hingga sistematika pembahasan.

    \item Studi Literatur

    Selanjutnya, Bab II  menjelaskan hasil studi literatur yang berkaitan dengan pengerjaan tugas akhir ini. Bab II ini berisi tentang pemahaman dasar seputar topik yang  dibahas pada tugas akhir ini.

    \item Analisis Persoalan dan Rancangan Solusi

    Pada Bab III  dijelaskan analisis persoalan untuk menyusun rancangan solusi. Rancangan tersebut  dijelaskan pada bab ini sebelum diimplementasikan. Rancangan solusi  dipaparkan dalam bentuk diagram dan kajian dalam bab ini.

    \item Implementasi dan Pengujian

    Bab IV ini  berisikan kajian terhadap implementasi yang telah dibuat. Bab ini juga  membahas tahap-tahap pengujian dan hasilnya. Perbandingan beberapa model prediksi  dibahas pada bab ini.

    \item Kesimpulan dan Saran

    Bab V  menutup tugas akhir ini. Konten pada bab ini  menjawab rumusan masalah. Bab ini juga  menyebutkan saran-saran perbaikan yang bisa dipakai untuk penelitian berikutnya. Bab ini  menyimpulkan hasil implementasi dan rancangan solusi terhadap masalah yang sudah diidentifikasi.
\end{enumerate}
