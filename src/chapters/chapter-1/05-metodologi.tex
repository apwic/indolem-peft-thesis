\section{Metodologi}

Terdapat metodologi yang digunakan untuk melaksanakan tugas akhir ini, berikut adalah tahapan pelaksanaan.

\begin{enumerate}
    \item Pengumpulan dataset dan persiapan data untuk berbagai tugas bahasa yang akan digunakan dalam eksperimen.
    \item \textit{Pre-training} model IndoBERT pada korpus teks bahasa Indonesia yang besar.
    \item Implementasi berbagai teknik \textit{transfer learning}, seperti LoRA, \textit{Prefix-Tuning}, \textit{Tiny-Attention Adapter}, dan \textit{Unified Parameter Transfer Learning}, pada model IndoBERT yang telah di-\textit{pretrain}.
    \item \textit{Fine-tuning} model-model hasil \textit{transfer learning} pada berbagai tugas bahasa yang telah disiapkan.
    \item Evaluasi kinerja model-model yang telah di-\textit{fine-tuning} menggunakan metrik yang sesuai untuk masing-masing tugas.
    \item Perbandingan hasil dan efisiensi penggunaan sumber daya antara teknik \textit{transfer learning} dan \textit{fine-tuning} tradisional.
\end{enumerate}