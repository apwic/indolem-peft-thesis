\section{Metodologi}

Terdapat metodologi yang digunakan untuk melaksanakan tugas akhir ini, berikut adalah tahapan pelaksanaannya.

\begin{enumerate}
    \item Identifikasi permasalahan

    Tahap awal penelitian ini adalah identifikasi permasalahan yang perlu dipecahkan. Permasalahan yang diidentifikasi berkaitan dengan pemanfaatan berbagai metode PEFT pada kakas evaluasi IndoLEM.

    \item Perancangan solusi permasalahan

    Dari masalah yang telah diidentifikasi, pada tahap ini dilakukan perancangan solusi permasalahan. Solusi yang ditelusuri merupakan implementasi metode \PEFT pada IndoLEM.

    \item Implementasi setiap metode PEFT

    Pada tahap ini dilakukan implementasi dari setiap metode \PEFT pada kakas IndoLEM untuk bisa dilakukan eksperimen.

    \item Eksperimen dengan setiap metode PEFT

    Pada tahap ini dilakukan eksperimen pada kakas IndoLEM yang sudah diaplikasikan metode \PEFT. Eksperimen  dijalankan dengan \textit{fine-tuning}  dan \PEFT.

    \item Evaluasi dan penarikan kesimpulan

    Pada tahap ini, dilakukan evaluasi dengan membandingkan hasil \textit{fine-tuning}  dengan \PEFT. Hasil evaluasi ini kemudian  menjadi tolak ukur terselesaikannya permasalahan pada tugas akhir ini.

\end{enumerate}
