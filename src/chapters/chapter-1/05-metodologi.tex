\section{Metodologi}

Terdapat metodologi yang digunakan untuk melaksanakan tugas akhir ini, berikut adalah tahapan pelaksanaannya.

\begin{enumerate}
    \item Identifikasi permasalahan
    
    Tahap awal penelitian ini adalah identikasi permasalahan yang perlu dipecahkan. Permasalahan yang diidentifikasi berkaitan dengan peningkatan kinerja model bahasa IndoBERT dalam berbagai tugas pemrosesan bahasa alami.

    \item Perancangan solusi permasalahan
    
    Dari masalah yang telah diidentifikasi, pada tahap ini dilakukan perancangan solusi permasalahan. Solusi yang ditelusuri merupakan penggunaan metode \PEFT pada IndoBERT.

    \item Implementasi setiap metode PEFT
    
    Pada tahap ini dilakukan implementasi dari setiap metode \PEFT pada kakas IndoLEM untuk bisa dilakukan eksperimen.

    \item Eksperimen dengan setiap metode PEFT
    
    Pada tahap ini dilakukan eksperimen pada kakas IndoLEM yang sudah menggunakan metode \PEFT. Eksperimen akan dijalankan dengan \textit{fine-tuning} tradisional dan \PEFT dengan 

    \item Evaluasi dan penarikan kesimpulan
    Pada tahap ini, dilakukan evaluasi dengan membandingkan hasil \textit{fine-tuning} tradisional dengan \PEFT. Hasil evaluasi ini kemudian akan menjadi tolak ukur terselesaikannya permasalahan pada tugas akhir ini.

\end{enumerate}
