\section{Metodologi}

Terdapat metodologi yang digunakan untuk melaksanakan tugas akhir ini, berikut adalah tahapan pelaksanaan.

\begin{enumerate}
    \item Identifikasi permasalahan
    
    Tahap awal penelitian ini adalah mengidentifikasi permasalahan yang perlu dipecahkan. Permasalahan yang diidentifikasi berkaitan dengan peningkatan kinerja model bahasa IndoBERT dalam berbagai tugas pemrosesan bahasa alami.

    \item Perancangan solusi permasalahan
    
    Setelah permasalahan diidentifikasi, langkah selanjutnya adalah merancang solusi untuk meningkatkan kinerja IndoBERT. Solusi yang ditelusuri berkaitan dengan penerapan berbagai teknik \textit{parameter-efficient transfer learning} seperti LoRA, \textit{Prefix-Tuning}, \textit{Tiny-Attention Adapter}, dan \textit{Unified View of Parameter-Efficient Transfer Learning}.

    \item Persiapan eksperimen
    
    Persiapan eksperimen meliputi dua aspek utama yaitu persiapan \textit{dataset} dan konfigurasi eksperimen.

    \item Persiapan \textit{dataset}
    
    Proses persiapan dataset melibatkan pengumpulan dan pemrosesan data yang akan digunakan dalam eksperimen. Data ini akan digunakan sebagai data \textit{training} dan data pengujian untuk model IndoBERT yang telah ditingkatkan.

    \item Implementasi dan eksperimen
    
    Tahap ini mencakup implementasi solusi yang telah dirancang dalam lingkungan eksperimen. Implementasi melibatkan praproses dataset, konfigurasi model, serta pengujian model yang telah ditingkatkan dengan berbagai teknik \textit{paramater-efficient transfer learning}.

    \item Analisis Hasil
    
    Setelah eksperimen selesai, hasil yang diperoleh akan dianalisis. Analisis meliputi perbandingan kinerja model terhadap berbagai tugas pemrosesan bahasa alami. Selain itu, analisis eror pada hasil prediksi juga akan dilakukan.

    \item Penarikan kesimpulan
    
    Kesimpulan dari penelitian ini akan menjawab rumusan masalah yang telah diajukan. Hasil penelitian akan menentukan apakah penerapan teknik \textit{paramater-efficient transfer learning} berhasil meningkatkan kinerja model IndoBERT dalam berbagai tugas pemrosesan bahasa alami.

\end{enumerate}