\section{Jadwal Pelaksanaan}
Jadwal pelaksanaan Tugas Akhir per minggunya dirincikan sebagai berikut:

\begin{landscape}
\begin{table}
\centering
\begin{ganttchart}[
    hgrid,
    vgrid={*{6}{draw=none}, dotted}, % Vertical lines only for weeks
    y unit title=0.5cm,
    y unit chart=0.7cm,
    x unit=5mm, % Adjust the horizontal scale
    time slot format=isodate,
    title height=1,
    title label font=\scriptsize, % Small font for titles
    bar/.append style={fill=blue!50},
    bar height=0.7,
    bar label font=\scriptsize, % Small font for bar labels
    inline % Place labels inline with bars
    ]{2023-09-18}{2023-11-03}
    \gantttitlecalendar{month=name, week} \\
    \ganttbar[inline=false]{Usulan Topik}{2023-09-18}{2023-09-22}\\
    \ganttbar[inline=false]{Eksplorasi Riset}{2023-09-25}{2023-09-29}\\
    \ganttbar[inline=false]{Metode \& Bab 2}{2023-10-02}{2023-10-06}\\
    \ganttbar[inline=false]{Prototipe Fine-Tuning}{2023-10-16}{2023-10-20}\\
    \ganttbar[inline=false]{Prototipe LoRA \& Dataset}{2023-10-23}{2023-10-27}\\
    \ganttbar[inline=false]{Bab 1 \& Prefix-Tuning}{2023-10-30}{2023-11-03}
\end{ganttchart}
\caption{Jadwal Pelaksanaan Tugas Akhir}
\label{ganttchartlabel}
\end{table}
\end{landscape}