\section{Latar Belakang}

Pengembangan model bahasa yang efisien dan akurat telah menjadi fokus utama dalam pengolahan bahasa alami. Salah satu langkah penting dalam pengembangan model bahasa adalah pre-training, di mana model dilatih pada korpus teks yang besar untuk memahami bahasa secara umum. Saat ini, model bahasa \textit{pre-trained} telah mencapai performa yang mengesankan dalam berbagai tugas pemrosesan bahasa alami (\textit{Natural Language Processing} - NLP).

IndoBERT adalah salah satu model bahasa \textit{pre-trained} yang telah berhasil diperkenalkan dalam konteks bahasa Indonesia. Meskipun IndoBERT telah mendapatkan pengakuan sebagai model terkemuka dalam berbagai tugas bahasa, seperti \textit{text classification},  \textit{named entity recognition} (NER), dan \textit{sentiment analysis}, masih ada potensi untuk meningkatkan performanya. Peningkatan performa model bahasa \textit{pre-trained} menjadi sangat penting dalam memastikan bahwa model dapat mengatasi bahasa yang kompleks dan variasi bahasa yang kaya dalam konteks pengolahan bahasa Indonesia.

Penggunaan metode transfer learning dalam \textit{fine-tuning} model bahasa telah menjadi pendekatan yang populer untuk meningkatkan kinerja model \textit{pre-trained}. Beberapa metode transfer learning yang efisien dari segi parameter telah muncul,  seperti LoRA, \textit{Prefix-Tuning}, \textit{Tiny-Attention Adapter}, dan \textit{Unified Parameter Transfer Learning}. Metode-metode ini memungkinkan peningkatan performa model \textit{pre-trained} tanpa memerlukan sumber daya sebanyak metode \textit{fine-tuning} tradisional. Oleh karena itu, penelitian ini akan menjelajahi berbagai teknik transfer learning ini untuk meningkatkan performa IndoBERT dalam berbagai tugas bahasa.

Dengan pemahaman yang lebih mendalam tentang teknik transfer learning yang dapat meningkatkan performa model bahasa \textit{pre-trained} dalam konteks bahasa Indonesia, penulis berharap penelitian ini akan memberikan wawasan berharga bagi pengembang model NLP dan praktisi di bidang pengolahan bahasa alami.
