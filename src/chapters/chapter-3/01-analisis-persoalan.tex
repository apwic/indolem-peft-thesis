\section{Analisis Persoalan}
\label{sec:analisis-persoalan}

Berdasarkan latar belakang yang telah diuraikan pada subbab \ref{sec:latar-belakang}, diperlukan implementasi metode \PEFT pada kakas IndoLEM. Untuk melakukan implementasi tersebut serta menjawab tujuan yang telah disebutkan pada subbab \ref{sec:tujuan}. Terdapat beberapa persoalan yang perlu dijawab.

\subsection{Kakas IndoLEM yang \textit{Outdated}}

Penelitian IndoLEM sudah diterbitkan semenjak 4 tahun yang lalu, tentunya banyak kakas yang sudah berkembang semenjak saat itu. Pada \textit{source code}-nya, terdapat \textit{script} untuk menjalankan pelatihan dan melakukan evaluasi. Tetapi, untuk menjalankannya membutuhkan konfigurasi dari Python, serta pustaka yang lain. Ini merupakan salah satu kelemahan dari kakas IndoLEM, yaitu kurangnya dokumentasi pada versi yang digunakan. Sebagai contoh, dari 7 tugas evaluasi, hanya terdapat 1 tugas evaluasi yang memberikan \textit{requirements} untuk menjalankan \textit{script} tersebut. Sehingga, banyak kesulitan yang muncul saat mencoba menjalankan proses pelatihan dan evaluasinya.

Selain itu, banyak pustaka yang digunakan seperti Torch, Transformers, dan Seqeval mempunyai versi yang lama. Sedangkan, untuk mengimplementasikan metode PEFT pada kakas IndoLEM memerlukan versi yang sesuai. Terdapat beberapa fungsi yang digunakan pada kakas IndoLEM sudah \textit{deprecated} karena sudah ada versi yang lebih baru dari fungsi tersebut.

Setiap tugas evaluasi pada IndoLEM mempunyai implementasi cara pelatihan dan evaluasi yang berbeda antara satu sama lain. Padahal, dengan menggunakan pustaka terbaru, sebagai contoh dengan menggunakan Trainer dari pustaka Huggingface, proses pelatihan dapat diimplementasikan secara konsisten. Sehingga, implementasi pada setiap task tidak perlu dibuat secara berbeda. Dari banyak alasan tersebut, perlu adanya refaktorisasi pada kakas IndoLEM.

\subsection{Implementasi Metode PEFT pada Kakas IndoLEM}

Kakas IndoLEM saat ini menggunakan metode \textit{fine-tuning} dalam proses pelatihan dan evaluasinya. Sedangkan, tujuan dari tugas akhir ini adalah untuk memanfaatkan metode PEFT pada kakas IndoLEM. Sehingga, perlu dilakukan implementasi dari setiap metode PEFT yang ingin digunakan agar bisa dimanfaatkan pada kakas IndoLEM.

Implementasi dari metode PEFT sudah dijelaskan pada penelitian dari metode PEFT tersebut. Terdapat beberapa kakas yang mengimplementasikan metode PEFT tersebut, sehingga bisa diintegrasikan pada proses pelatihan dan evaluasinya. Dari kakas yang tersedia, ada yang hanya mengimplementasikan satu metode, beberapa metode, atau bahkan ada yang menggabungkan beberapa metode tersebut. Berdasarkan hal ini, untuk mengimplementasikan metode PEFT pada kakas IndoLEM, diperlukan pemilihan kakas yang tepat serta integrasinya.

\subsection{Pemilihan Model}

Tugas evaluasi yang digunakan pada tugas akhir ini, sesuai yang telah disebutkan pada subbab \ref{sec:batasan-masalah}, adalah \nlptask. Setiap tugas evaluasi mempunyai kategori yang berbeda. \textit{Named entity recognition} merupakan tugas \textit{token classification}, yang berarti setiap token perlu diklasifikasikan terhadap sebuah label. \textit{Sentiment analysis} merupakan tugas \textit{text classification},  yang berarti suatu atau beberapa kalimat diklasifikan terhadap suatu sentimen (label) tertentu. Terakhir, tugas \textit{summarization} merupakan tugas \textit{generation} yang berarti dari suatu teks perlu diolah menjadi tesk lain.

Setiap model mempunyai karakteristik yang berbeda, sebagai contoh pada arsitektur Transformers, ada model yang merupakan \textit{encoder}, \textit{decoder}, atau \textit{encoder decoder}. Tugas \textit{classification} seperti NER dan \textit{sentiment analysis} hanya memerlukan model \textit{encoder}, sedangkan tugas \textit{generation} memerlukan \textit{encoder decoder}. Untuk melakukan pelatihan dan evaluasi terhadap kakas IndoLEM diperlukan pemilihan model yang sesuai.

\subsection{Pelatihan dan Evaluasi Model}

Untuk menguji metode PEFT yang nanti akan diimplementasikan, perlu adanya pelatihan dan evaluasi pada model. Pelatihan dan evaluasi ini perlu dilakukan agar kakas evaluasi dapat memberikan metriks evaluasi yang menjadi hasil dari pelatihan model tersebut. Pelatihan dan evaluasi model memerlukan beberapa komponen, yaitu \textit{dataset} (latih, evaluasi, dan uji), \textit{hyperparameter} yang digunakan, dan lingkungan pelatihan.

Setiap komponen tersebut perlu ditentukan agar proses pelatihan dan evaluasi model dapat berjalan dengan sesuai. Untuk setiap pelatihan dan evaluasi model, dilakukan metode \textit{fine-tuning} serta berbagai metode PEFT. Hal ini perlu dilakukan agar setiap metode tersebut bisa dibandingkan antara satu sama lain.


Berdasarkan latar belakang yang telah diuraikan pada subbab \ref{sec:latar-belakang}, peningkatan kinerja model IndoBERT dilakukan dengan menggunkan metode \PEFT. Peningkatan kinerja model dengan metode PEFT ini melibatkan penambahan konfigurasi ataupun \textit{layer} tambahan terhadap model tergantung dengan karateristik dari metode PEFT masing-masing. Selain itu, analisis komprehensif dibutuhkan untuk membandingkan metode PEFT yang dapat memberikan hasil terbaik. Untuk membuat analisis yang komprehensif dibutuhkan metode evaluasi terhadap kinerja model. Secara keseluruhan, penelitian yang dilakukan pada tugas akhir ini mencakup tiga tahap sebagai berikut.

