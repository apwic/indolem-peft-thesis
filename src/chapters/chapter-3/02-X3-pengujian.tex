\subsection{Peningkatan Kinerja Model IndoBERT}

Peningkatan kinerja model IndoBERT merupakan tahap terakhir dari yang sebelumnya telah dilakukan, yaitu refaktorisasi dan pengembangan metode PEFT. Pada tahapan ini  dijalankan eksperimen untuk memvalidasi peningkatan kinerja dari model. Prosesnya terdiri dari praproses data, \textit{hyperparameter tuning}, \textit{training}, dan evaluasi. 

% TODO: tambahin yang udah dilakuin di paper asli IndoBERT
\textit{Dataset} sudah tersedia pada kakas IndoLEM, sehingga bisa langsung digunakan. Tetapi, perlu dilakukan praproses terlebih dahulu. Pada proses refaktorisasi sebelumnya, praproses data sudah termasuk ke dalamnya. Jadi, pada tahap ini praproses data bisa langsung dijalankan.

\textit{Hyperparameter tuning} diperlukan untuk menentukan \textit{hyperparameter} terbaik pada proses pelatihan. Pada proses ini, terdapat dua eksperimen yang dilakukan pada model, yaitu tuning jumlah \textit{epoch} dan/atau jumlah \textit{steps} serta \textit{learning rate}. \textit{Hyperparameter tuning} hanya dilakukan pada \textit{tuning} dengan metode PEFT karena pada \textit{fine-tuning} tradisional  mengacu pada \textit{paper} aslinya.

Proses \textit{training}  dilakukan sebanyak jumlah metode PEFT yang digunakan ditambah dengan satu yaitu \textit{fine-tuning} tradisional. \textit{Training}  memakai data yang sudah dipraproses dan \textit{hyperparameter} yang telah dilakukan \textit{hyperparameter tuning}. Selanjutnya hasil dari proses \textit{training}  dievaluasi.

Proses evaluasi  memabandingkan hasil prediksi dengan \textit{ground truth} untuk menentukan skornya. Metriks evaluasi ini  berbeda untuk setiap tugas NLP, sebagai contoh untuk tugas NER dan sentimen  digunakan F1. Selain itu, evaluasi  dilakukan pada setiap metode PEFT untuk menentukan jumlah parameter yang digunakan dan juga sumber daya komputasinya. 
