\section{Metriks Evaluasi}

Dalam proses \textit{training} model, diperlukan metode evaluasi sebagai penilaian kinerja model dengan membandingkan metriks yang dihasilkan oleh model tersebut. Metriks ini digunakan pada model klasifikasi dengan membandingkan hasil prediksi dengan target aslinya. Hasil prediksi ini dapat direpresentasikan sebagai \textit{confusion matrix}. Berdasarkan \citeauthor{metrics}, terdapat beberapa metriks yang dapat dihasilkan berdasarkan \textit{confusion matrix}, yaitu \textit{accuracy}, \textit{precision}, \textit{recall}, dan F1-\textit{Score}.

\begin{table}[ht]
    \centering
        \caption{\textit{Confusion matrix}}
        \label{table:confusion-matrix}
        \begin{tabular}{|l|c|c|}
        \hline
        \rowcolor{black!10}
        \multicolumn{1}{|c|}{\textbf{Target}} & \multicolumn{2}{c|}{\textbf{Prediksi}} \\ \cline{2-3} 
        \rowcolor{black!10}
        & \textbf{Positif} & \textbf{Negatif} \\ \hline
        \textbf{Positif} & \textit{True Positive} (TP) & \textit{False Negative} (FN) \\ \hline
        \textbf{Negatif} & \textit{False Positive} (FP) & \textit{True Negative} (TN) \\ \hline
    \end{tabular}
\end{table}

Berdasarkan tabel \ref{table:confusion-matrix} terdapat empat hasil prediksi. \textit{True Positive} (TP) yang berarti model memprediksi kelas positif yang memang bernilai positif sesuai kelasnya. True Negative (TN) memprediksi kelas negatif dan kelas tersebut memang bernilai negatif. Sedangkan, \textit{False Positive} (FP) me\textit{mprediksi kelas} positif, tetapi kelas tersebut bernilai negatif. Sebaliknya juga untuk \textit{False Negative} (FN) model memprediksi kelas tersebut sebagai negatif, tetapi kelas tersebut bernilai positif.

\subsection{\textit{Accuracy}}
Nilai \textit{accuracy} dihitung sebagai proporsi hasil prediksi yang benar (baik TP maupun TN) dari total jumlah prediksi. Metriks ini memberikan gambaran keseluruhan tentang keefektifan model. Dengan rumus sebagai berikut.

\begin{equation}
    Accuracy = \frac{TP + TN}{TP + TN + FP + FN}
\end{equation}

\subsection{\textit{Precision}}
Nilai \textit{precision} menilai proporsi prediksi positif yang benar-benar positif. Metriks ini memberikan gambaran tentang keakuratan model dalam memprediksi kelas positif. Dengan rumus sebagai berikut.
\begin{equation}
    Precision = \frac{TP}{TP + FP}
\end{equation}

\subsection{\textit{Recall}}
Nilai \textit{recall} menilai proporsi kasus positif sebenarnya yang berhasil diidentifikasi oleh model. Metrix Ini berfungsi untuk mengukur kemampuan model dalam mengidentifikasi semua kasus positif. Dengan rumus sebagai berikut.
\begin{equation}
    Recall = \frac{TP}{TP + FN}
\end{equation}

\subsection{F1-\textit{Score}}
Nilai F1-\textit{Score} direpresentasikan sebagai nilai rata-rata harmonis dari nilai \textit{precision} dan \textit{recall}. Metriks ini berfungsi untuk mengukur keseimbangan dari kedua metriks tersebut. Dengan rumus sebagai berikut.
\begin{equation}
    F1-Score = 2 \times \frac{Precision \times Recall}{Precision + Recall}
\end{equation}
