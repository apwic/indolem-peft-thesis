\subsection{IndoLEM}

IndoLEM muncul sebagai respons terhadap kebutuhan industri dan komunitas penelitian untuk memiliki \textit{benchmark} yang khusus dirancang untuk mengevaluasi model NLP dalam konteks bahasa Indonesia \parencite{indobert}. Meskipun ada banyak \textit{benchmark} NLP yang tersedia, seperti GLUE, kebanyakan dari mereka berfokus pada bahasa Inggris. Namun, dengan keragaman linguistik yang kaya dan jumlah penutur yang besar, bahasa Indonesia memerlukan pendekatan khusus dalam evaluasi model NLP.

IndoLEM tidak hanya menyediakan kumpulan tugas evaluasi yang dirancang khusus untuk bahasa Indonesia, tetapi juga memastikan bahwa tugas-tugas tersebut mencerminkan nuansa dan tantangan unik yang diasosiasikan dengan bahasa ini. Misalnya, tugas analisis sentimen dalam IndoLEM mungkin melibatkan pengolahan ulasan produk dalam bahasa Indonesia, sementara tugas klasifikasi teks mungkin berfokus pada kategorisasi artikel berita lokal.

Salah satu aspek penting dari IndoLEM adalah dataset yang digunakannya. Mengingat pentingnya data dalam \textit{training} dan evaluasi model NLP, IndoLEM memastikan bahwa dataset yang digunakan berasal dari sumber-sumber lokal yang relevan. Ini memastikan bahwa model yang dievaluasi dengan IndoLEM benar-benar diuji dalam konteks yang sesuai dengan penggunaan sebenarnya dalam kehidupan nyata.