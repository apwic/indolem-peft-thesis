\section{\textit{Natural Language Processing}}

Pemrosesan Bahasa Alami (PBA) atau dalam bahasa Inggris dikenal dengan \textit{Natural Language Processing} (NLP) merupakan cabang dari ilmu komputer, kecerdasan buatan, dan linguistik yang berfokus pada interaksi antara komputer dan bahasa manusia. NLP bertujuan untuk memungkinkan komputer tidak hanya memahami dan menafsirkan bahasa manusia, tetapi juga untuk menghasilkannya dengan cara yang bermakna dan efektif. Hal ini dijelaskan oleh \citeauthor{nlp} \parencite{nlp}, menyatakan pentingnya NLP dalam membangun jembatan komunikasi antara manusia dan mesin.

Dalam beberapa dekade terakhir, NLP telah mengalami kemajuan yang signifikan, memungkinkan komputer tidak hanya memahami bahasa manusia tetapi juga merespons dengan cara yang semakin kompleks dan kontekstual. Teknologi seperti mesin penerjemah, asisten virtual, dan sistem rekomendasi semuanya memanfaatkan prinsip-prinsip NLP untuk berfungsi.

Salah satu tantangan utama dalam NLP adalah keragaman dan kompleksitas bahasa manusia. Bahasa penuh dengan nuansa, ambiguitas, dan struktur yang dapat bervariasi tergantung pada konteks dan budaya. Untuk mengatasi tantangan ini, berbagai tugas NLP telah didefinisikan dan dikembangkan untuk memecah masalah pemahaman bahasa menjadi komponen yang lebih kecil dan lebih spesifik.

Beberapa tugas NLP yang umum antara lain \textit{Part of Speech} (POS) \textit{Tagging}, \textit{Named Entity Recognition} (NER), \textit{Dependency Parsing}, \textit{Sentiment Analysis}, dan \textit{Summarization}. Masing-masing tugas ini menargetkan aspek tertentu dari pemahaman bahasa dan memiliki aplikasi praktisnya sendiri dalam berbagai bidang, mulai dari analisis teks hingga pengembangan sistem percakapan otomatis.
