\subsection{\textit{General Language Understanding Evaluation} (GLUE)}

GLUE, yang merupakan singkatan dari \textit{General Language Understanding Evaluation}, adalah inisiatif yang dirancang untuk memajukan pemahaman bahasa alami melalui evaluasi yang konsisten dan komprehensif \parencite{glue}. Dengan latar belakang yang semakin meningkatnya kompleksitas dan variasi model NLP, muncul kebutuhan untuk memiliki metrik evaluasi yang standar dan konsisten. 

Salah satu ciri khas dari GLUE adalah kumpulan tugas evaluasinya yang beragam. Ini mencakup tugas-tugas seperti analisis sentimen, di mana tujuannya adalah untuk menentukan apakah teks tertentu memiliki konotasi positif, negatif, atau netral; jawaban pertanyaan, di mana model diberi pertanyaan dan harus memilih jawaban yang paling tepat dari kumpulan pilihan; dan entailment teks, di mana model harus menentukan apakah satu kalimat secara logis mengikuti kalimat lain.

Namun, bukan hanya variasi tugas yang membuat GLUE menjadi penting. GLUE juga menyediakan papan peringkat, yang memungkinkan peneliti dari seluruh dunia untuk membandingkan kinerja model mereka dengan model-model lain dalam kondisi yang sama. Ini menciptakan lingkungan kompetitif yang sehat, di mana peneliti didorong untuk terus meningkatkan teknik dan metode mereka untuk menduduki peringkat teratas.

Selain itu, GLUE juga memberikan kesempatan bagi peneliti untuk memahami kelemahan dan kekuatan model mereka. Dengan memiliki berbagai tugas evaluasi, peneliti dapat melihat di mana model mereka unggul dan di mana mereka mungkin memerlukan perbaikan. Ini sangat berharga dalam pengembangan model NLP yang lebih canggih dan efisien.