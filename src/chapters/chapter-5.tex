\chapter{Penutup}

Bab Kesimpulan dan Saran  menjadi bagian akhir dan penutup dari penelitian tugas akhir ini. Bab ini  membahas kesimpulan yang berisi ketercapaian tujuan penelitian tugas akhir dengan permasalahan yang diselesaikan dalam penelitian tugas akhir. Selain itu, bab ini  membahas saran yang dapat dilakukan untuk pengembangan atau penelitian selanjutnya:

\section{Kesimpulan}

Tugas akhir ini memanfaatkan metode \PEFT pada kakas evaluasi IndoLEM. Dalam prosesnya, diperlukan implementasi metode PEFT terhadap IndoLEM. Metode PEFT yang dimaanfatkan pada tugas akhir ini adalah \methodPEFT. Lalu, dilakukan eksperimen dan evaluasi terhadap eksperimen tersebut. Eksperimen dan evaluasi dilakukan pada tiga tugas evaluasi yaitu \nlptask. Terdapat beberapa kesimpulan yang dapat diambil, yaitu sebagai berikut.

\begin{enumerate}
    \item{
        Implementasi metode PEFT pada kakas evaluasi IndoLEM dapat dilakukan. Refaktorisasi terhadap IndoLEM diperlukan untuk mendukung hal-hal yang diperlukan untuk memgimplementasikan metode PEFT. Selanjutnya, implementasi dilakukan dengan menggunakan kakas bantuan yaitu Transformers dan Adapters. Transformers digunakan untuk hal-hal terkait pelatihan dan evaluasi model dan Adapters digunakan untuk kebutuhan PEFT.
    }
    \item {
            Metode PEFT secara umum memberikan hasil yang dapat menyaingi \textit{fine-tuning} pada setiap tugas evaluasi. Kinerja yang dihasilkan dari setiap metode PEFT juga cukup variatif ketika dibandingkan antara tugas evaluasi. Pada tugas evaluasi \textit{classification}, yaitu NER dan \textit{sentiment analysis}, didapatkan hasil bahwa setiap metode PEFT mampu menyaingi kinerja \textit{fine-tuning} dengan rentang selisih antara -0,8\% sampai -5,4\% untuk tugas NER dan antara -1\% sampai -6,2\% untuk tugas \textit{sentiment analysis}. Perbandingan antara setiap metode PEFT mempunyai tren yang sama untuk tugas \textit{classification}, dengan urutan metode PEFT yang menghasilkan kinerja terbaik adalah \textit{Prefix-Tuning}, UniPELT, \textit{Adapter}, dan \textit{LoRA}. Untuk tugas \textit{summarization}, rentang selisih yang didapatkan dari perbandingan metode PEFT adalah -90\% sampai +1,25\%. LoRA dan \textit{Adapter} mampu menghasilkan kinerja yang lebih baik, namun \textit{Prefix-Tuning} dan UniPELT gagal mendapatkan kinerja yang menyaingi \textit{fine-tuning}.
    }
    \item {
            Parameter pelatihan dan waktu pelatihan yang digunakan oleh metode PEFT terbukti lebih efisien dibandingkan dengan \textit{fine-tuning}. Secara umum metode PEFT menggunakan paramater pelatihan yang lebih sedikit dengan rentang antara 0,2\% sampai 15\% dari parameter model. Untuk waktu pelatihan, metode LoRA dan \textit{Adapter} secara konsisten menggunakan waktu pelatihan yang lebih sedikit dari \textit{fine-tuning}. Sementara, \textit{Prefix-Tuning} hanya gagal pada tugas \textit{summarization}. Lalu, UniPELT hanya mampu memberikan waktu pelatihan yang lebih sedikit pada tugas \textit{sentiment analysis}.
    }
    \item {
            Metode PEFT yang memberikan hasil paling konsisten dalam hal kinerja, parameter pelatihan, dan waktu pelatihan adalah LoRA dan \textit{Adapter}. Hasil yang didapatkan dapat menyaingi bahkan lebih baik dibandingkan \textit{fine-tuning}, dengan parameter pelatihan dan waktu pelatihan yang konsisten lebih sedikit. Prefix-Tuning berhasil pada tugas \textit{classification}, namun gagal pada tugas \textit{generation}. Terakhir, UniPELT menggunakan parameter paling banyak dibandingkan metode lain, namun menggunakan waktu pelatihan yang kadang lebih lama dibanding \textit{fine-tuning}.
    }
\end{enumerate}

\section{Saran}

Tugas akhir ini tentunya tidak sempurna dan terdapat banyak kekurangannya. Sehingga, penelitian atau pengembangan lebih lanjut tentunya dapat memperbaiki kekurangan tersebut. Berikut saran-saran yang dapat dilakukan untuk penelitian selanjutnya:
\begin{enumerate}
    \item{
        Metode PEFT yang dimaanfatkan pada tugas akhir ini terbatas, sehingga dapat dimaanfatkan lebih banyak lagi metode PEFT untuk kakas IndoLEM.
    }
    \item {
        Tugas evaluasi pada kakas IndoLEM hanya digunakan tiga, yaitu \nlptask. Penelitian selanjutnya, dapat menggunakan lebih banyak lagi tugas evaluasi atau bahkan dapat ditambahkan dari yang tidak disediakan pada kakas IndoLEM.
    }
    \item {
        Pada tugas akhir ini, tidak dilakukan evaluasi terhadap pustaka yang digunakan, terutama pustaka implementasi PEFT, yaitu Adapters. Untuk penelitian selanjutnya, diharapkan dapat mencoba dan mengevaluasi lebih banyak lagi pustaka PEFT untuk membandingkan hasil yang didapatkan dari tugas akhir ini.
    }
\end{enumerate}
