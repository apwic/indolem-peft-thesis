\clearpage
\chapter*{DAFTAR ISTILAH DAN SINGKATAN}
\addcontentsline{toc}{chapter}{DAFTAR ISTILAH DAN SINGKATAN}
\begin{center}
   \begin{table}[h]
        \renewcommand{\arraystretch}{1.5}
        \begin{tabularx}{\textwidth}{lXr}
            \textbf{Istilah/Singkatan} & \textbf{Keterangan} & \textbf{Halaman} \\
                                       & & \textbf{kemunculan} \\
                                       & & \textbf{pertama} \\
            \textit{Adapter} & Metode PEFT dengan menambahkan modul \textit{adapter} yang berisi matriks proyeksi dan fungsi nonlinear & 2 \\
            BERT & \textit{Bidirectional Encoder Representations from Transformers} : Model \textit{encoder} dengan arsitektur \textit{Transformer} yang memproses teks secara \textit{bidirectional} & 11 \\
            IndoLEM & Kakas evaluasi untuk NLU dalam bahasa Indonesia & 1 \\
            LoRA & \textit{Low Rank Adaptation} : metode PEFT yang menggunakan matriks dimensi rendah pada proses pelatihan model & 2 \\
            NLP & \textit{Natural Language Processing} : suatu bidang pada AI yang terfokus pada pengolahan bahasa alami & 1 \\
            NER & \textit{Named Entity Recoginition} : tugas NLP untuk menentukan jenis entitas dari setiap token & 2 \\
            NLU & \textit{Natural Language Understanding} : pemahaman pada bahasa alami & 1 \\
            NLG & \textit{Natural Language Generation} : tugas generasi pada NLP & 1 \\
            PEFT & \textit{Parameter-Efficient Fine-Tuning} : variasi dari metode \textit{fine-tuning} dengan menggunakan parameter yang lebih sedikit (efisien) & 2 \\
            \textit{Prefix-Tuning} & metode PEFT yang menambahkan \textit{prefix} sebagai konteks tambahan pada proses pelatihan model & 2 \\
        \end{tabularx}
    \end{table}
\end{center}
\clearpage
\begin{center}
   \begin{table}[t!]
        \renewcommand{\arraystretch}{1.5}
        \begin{tabularx}{\textwidth}{lXr}
            \textbf{Istilah/Singkatan} & \textbf{Keterangan} & \textbf{Halaman} \\
                                       & & \textbf{kemunculan} \\
                                       & & \textbf{pertama} \\
            ROUGE & \textit{Recall-Oriented Understudy for Gisting Evaluation} : Metrik evaluasi dengan mengukur kemampuan model dalam memprediksi kata-kata yang muncul pada target & 29 \\
            T5 & \textit{Text-to-Text Transfer Transformers} : Model \textit{encoder decoder} dengan arsitektur \textit{Transformers} yang memproses dari teks menjadi teks lagi & 13 \\
            \textit{Transformers} & Arsitektur \textit{neural network} yang mempunyai mekanisme \textit{attention} untuk dapat fokus terhadap suatu konteks & 9 \\
        \end{tabularx}
    \end{table}
\end{center}
