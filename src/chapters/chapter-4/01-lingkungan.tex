\section{Lingkungan Eksperimen}
\label{sec:lingkungan-eksperimen}

Lingkungan eksperimen yang dipakai berbasis \textit{cloud} dengan menggunakan GPU khusus untuk pelatihan model, sehingga lingkungan ini khusus digunakan untuk melakukan pelatihan dan evaluasi model. Lingkungan ini dibagi menjadi dua bagian, yaitu untuk tugas evaluasi \textit{classification} dan \textit{generation}. Pembagian ini didasarkan pada kebutuhan memori yang berbeda antara kedua tugas tersebut. Tugas evaluasi \textit{generation} membutuhkan memori pada GPU yang lebih besar karena perlu memproses data dengan volume lebih banyak. Selain itu, tugas evaluasi \textit{generation} perlu melakukan generasi terhadap teks untuk menghasilkan evaluasi, sehingga membutuhkan memori yang lebih banyak.

\begin{table}[h]
    \vspace{0.25cm}
    \centering
    \caption{Lingkungan eksperimen}
    \label{table:lingkungan-eksperimen}
    \begin{tabular}{lcc}
        \toprule
        & \textbf{Tugas \textit{Classification}} & \textbf{Tugas \textit{Generation}} \\ 
        \midrule
        GPU          & Nvidia Tesla V100  & Nvidia A40       \\
        GPU RAM      & 16 GB              & 45 GB            \\
        TeraFLOPS    & 11.3               & 29.9             \\
        Lokasi       & Swedia             & Kroasia          \\
        Jumlah GPU   & 1                  & 1                \\
        \bottomrule
    \end{tabular}
\end{table}

Spesifikasi lengkap dari lingkungan eksperimen dapat dilihat pada tabel \ref{table:lingkungan-eksperimen}. Penggunaan dari lingkungan eksperimen tersebut adalah dengan mengakses GPU tersebut melalui SSH yang telah disediakan dari Vast.ai. \textit{Dataset} yang digunakan pada proses eksperimen akan dimuat pada GPU sehingga waktu pelatihan tidak terpengaruh oleh latensi yang disebabkan oleh lokasi GPU yang berbeda.

\begin{table}[h]
    \vspace{0.25cm}
    \centering
    \caption{Versi kakas dan perangkat lunak}
    \label{table:tools-version}
    \begin{tabular}{ll}
        \toprule
        \textbf{Kakas} & \textbf{Versi} \\ 
        \midrule
        Python                & 3.11        \\
        Transformers          & 4.40.2      \\
        Adapters              & 0.2.2       \\
        Datasets              & 2.20.0      \\
        Evaluate              & 0.4.2       \\
        Numpy                 & 2.0.0       \\
        Pandas                & 2.2.2       \\
        Seqeval               & 1.2.2       \\
        Torch                 & 2.3.1       \\
        Wandb                 & 0.17.3      \\
        nltk                  & 3.8.1       \\
        rouge\_score          & 0.1.2       \\
        huggingface\_hub      & 0.23.4      \\
        \bottomrule
    \end{tabular}
\end{table}

Versi dari kakas dan perangkat lunak yang digunakan dapat dilihat pada tabel \ref{table:tools-version}. Setiap eksperimen yang dilakukan menggunakan \textit{virtual environment} yang dibuat berdasarkan versi dari kakas dan perangkat lunak tersebut untuk memastikan konsistensi pada eksperimen. Lingkungan dibuat dengan menggunakan \textit{virtual environment} dari Conda.
