\clearpage
\chapter*{ABSTRAK}
\addcontentsline{toc}{chapter}{ABSTRAK}

\begin{center}
    \center
    \begin{singlespace}
        \large\bfseries\MakeUppercase{\thetitle}
    
        \normalfont\normalsize
        Oleh:
    
        \bfseries \theauthor
    \end{singlespace}
\end{center} 

\begin{singlespace}
    Metode \textit{fine-tuning} digunakan sebagai metode pelatihan untuk melakukan evaluasi pada berbagai evaluasi NLU. Tugas evaluasi IndoLEM yang merupakan pionir dari evaluasi NLU berbahasa Indonesia menggunakan \textit{fine-tuning} sebagai metode pelatihannya. \textit{Fine-tuning} melatih model dengan mengubah seluruh parameter model. Hal ini bisa menjadi tantangan dari segi memori dan juga waktu pelatihannya. Terdapat metode \PEFT yang dapat melatih model dengan kinerja yang sebanding dengan metode \textit{fine-tuning}. Dalam tugas akhir ini, berbagai metode \PEFT, yaitu \methodPEFT dimaanfaatkan dalam tugas evaluasi IndoLEM tersebut. Tujuan dari tugas akhir ini adalah untuk memanfaatkan metode PEFT pada IndoLEM. Pemanfaatan tersebut mencakup, penggunaan metode PEFT pada IndoLEM, perbandingkan kinerja terhadap setiap metode PEFT, dan analisis terkait penggunaan parameter dan waktu pelatihan.
    Tugas akhir ini berhasil memanfaatkan metode PEFT pada IndoLEM. Dengan dilakukan refaktorisasi terhadap IndoLEM, metode PEFT dapat dimanfaatkan. Selanjutnya, Eksperimen dilakukan dengan melatih model dengan metode \textit{fine-tuning} dan metode PEFT. Pengujian dilakukan pada 3 tugas evaluasi, yaitu \nlptask. Hasil eksperimen menunjukkan bahwa PEFT hanya menggunakan sekitar 0,2\% sampai 15\% dari parameter pelatihan model, dengan menggunakan waktu pelatihan yang lebih cepat. Kinerja yang didapatkan untuk tugas NER dan \textit{sentiment analysis} berkisar pada rentang -0,8\% sampai -6,2\%. Hal ini menunjukkan adanya \textit{trade off} antara penggunaan parameter pelatihan dengan kinerja yang dihasilkan. Namun, metode \textit{Prefix-Tuning} dan UniPELT gagal untuk memberikan hasil yang konsisten pada tugas \textit{summarization}.

    \textit{\textbf{Kata kunci: Fine-tuning, Parameter-efficient, IndoLEM, NER, Sentiment Analysis, Summarization}}
\end{singlespace}

\clearpage
