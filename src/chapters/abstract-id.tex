\clearpage
\chapter*{ABSTRAK}
\addcontentsline{toc}{chapter}{Abstract}

\begin{center}
    \center
    \begin{singlespace}
        \large\bfseries\MakeUppercase{\thetitle}
    
        \normalfont\normalsize
        By:
    
        \bfseries \theauthor
    \end{singlespace}
\end{center} 

\begin{singlespace}
    Metode \textit{fine-tuning} tradisional selalu digunakan sebagai metode pelalihan untuk melakukan evaluasi pada berbagai evaluasi NLU. Kakas evaluasi IndoLEM yang merupakan pionir dari evaluasi NLU berbahasa Indonesia menggunakan \textit{fine-tuning} sebagai metode pelatihannya. \textit{Fine-tuning} melatih model dengan mengubah seluruh paramter model. Hal ini bisa menjadi tantangan dari segi memori dan juga waktu pelatihannya. Terdapat metode \PEFT yang dapat melatih model dengan kinerja yang sebanding dengan metode \textit{fine-tuning} tradisional. Dalam tugas akhir ini, berbagai metode \PEFT, yaitu \methodPEFT dimaanfaatkan dalam kakas evaluasi IndoLEM tersebut.
\end{singlespace}

\clearpage
