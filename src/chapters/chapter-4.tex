\chapter{Implementasi, Eksperimen, dan Hasil Evaluasi Eksperimen}
Bab ini menjelaskan proses implementasi dari rancangan solusi yang telah dikaji pada Bab \ref{sec:chapter-3} dalam menyelesaikan permasalahan utama tugas akhir, yaitu pemanfaatan berbagai metode PEFT pada kakas evaluasi IndoLEM. Selain itu, dijelaskan juga lingkungan, implementasi, dan eksperimen.

Berdasarkan rancangan solusi yang diajukan pada subbab \ref{sec:rancangan-solusi}, pembangunan ulang kakas IndoLEM diperlukan untuk mengimplementasi setiap metode PEFT. Model yang dipilih yaitu IndoBERT untuk tugas evaluasi \textit{classification} dan IndoT5 untuk tugas evaluasi \textit{generation} dilatih pada lingkungan eksperimen. Model yang telah dilatih dengan metode \textit{fine-tuning} tradisional dan metode PEFT dievaluasi untuk membandingkan kinerja dari setiap metode pada setiap tugas evaluasinya.

\section{Lingkungan}
\section{Pengembangan Ulang Kakas IndoLEM}
\label{sec:pengembangan-kakas}

Pengembangan ulang kakas IndoLEM ini diperlukan untuk memperbarui dan memperbaiki infrastruktur dari kakas tersebut. Pengembangan ini dilakukan pada lingkungan lokal pribadi, untuk keperluan eksperimen dilakukan pada lingkungan \textit{cloud} seperti yang sudah disebutkan pada subbab \ref{sec:lingkungan-eksperimen}.

\subsection{Persiapan Lingkungan Pengembangan}

Implementasi diawali dengan menyiapkan lingkungan pengembangan, untuk menyiapkan lingkungan pengembangan dilakukan dengan melakukan instalasi terhadap perangkat lunak dan pustaka yang dibutuhkan. Berikut merupakan \textit{requirements.txt} yang berisi pustaka yang perlu dilakukan intalasi yang bisa dilihat pada tabel \ref{table:requirements}

\begin{table}[h]
    \caption{Tabel \textit{requirements.txt}}
    \label{table:requirements}
    \begin{lstlisting}[language=bash]
    accelerate
    adapters==0.2.2
    datasets
    evaluate
    huggingface-hub
    numpy
    pandas
    scikit-learn
    seqeval
    torch
    transformers==4.40.*
    wandb
    nltk
    rouge_score
    indobenchmark-toolkit
    \end{lstlisting}
\end{table}

Untuk memudahkan pengembangan perlu membuat \textit{virtual environment} untuk mengenkapsulasi pustaka yang dipakai. \textit{Virtual environment} bisa dibuat dengan menggunakan Conda atau pustaka virtualenv. Untuk tugas akhir ini, digunakan Conda. \textit{Command} yang digunakan dapat dilihat pada tabel \ref{table:command-virtualenv}.

\begin{table}[h]
    \caption{Tabel \textit{command virtual environment}}
    \label{table:command-virtualenv}
    \begin{lstlisting}[language=bash]
    # Membuat virtual environment
    conda create env -n indolem python=3.11
    # Menjalankan virtual environment
    conda activate indolem
    # Melakukan instalasi pustaka 
    pip install -r requirements.txt
    \end{lstlisting}
\end{table}

\textit{Command} tersebut akan membuat \textit{virtual environment}, mengakitfkannya, dan melakukan instalasi pada versi Python dan pustaka yang dibutuhkan.

\subsection{Penguraian Argumen}
\label{sec:parse-arg}

Penguraian argumen dilakukan dengan menggunakan modul HfArgumentParser dari pustaka Transformers. Modul ini digunakan untuk memudahkan penguraian argumen karena argumen yang dipakai untuk model, pelatihan, dan evaluasi sudah diimplementasikan dari modul tersebut. Untuk argumen tambahan yang spesifik pada suatu tugas evaluasi perlu ditambahkan secara manual, sebagai contoh pada tugas evaluasi NER terdapat argumen \textit{return\_entity\_level\_metrics} yaitu menggunakan metriks evaluasi untuk level entitasnya.

\begin{table}[h]
    \caption{Tabel kode HfArgumentParser}
    \label{table:hfargumentparser}
    \begin{lstlisting}[language=python]
    parser = HfArgumentParser(
        [
            ModelArguments,
            DataTrainingArguments,
            TrainingArguments,
            AdapterArguments,
            WandbArguments,
        ]
    )

    if len(sys.argv) == 2 and sys.argv[1].endswith(".json"):
        # If we pass only one argument to the script and it's the path to a json file,
        # let's parse it to get our arguments.
        model_args, data_args, training_args, adapter_args, wandb_args = (
            parser.parse_json_file(json_file=os.path.abspath(sys.argv[1]))
        )
    else:
        model_args, data_args, training_args, adapter_args, wandb_args = (
            parser.parse_args_into_dataclasses()
        )
    \end{lstlisting}
\end{table}

Berdasarkan tabel \ref{table:hfargumentparser} terdapat 5 jenis argumen, yaitu model, \textit{data training}, pelatihan, \textit{adapter}, dan wandb. Argumen model digunakan untuk pemuatan model. Argumen \textit{data training} digunakan untuk pemuatan \textit{file} (latih, evaluasi, dan uji), \textit{padding}, maksimum sampel, maksimum sekuens, dan lain sebagainya yang berhubungan dengan \textit{dataset}. Lalu, argumen pelatihan merupakan \textit{hyperparemeter} yang digunakan pada proses pelatihan seperti \textit{learning rate, epoch, batch size}, dan lain sebagainya. Selanjutnya, argumen \textit{adapter} berfungsi sebagai konfigurasi untuk jenis metode PEFT beserta \textit{hyperparameter} yang digunakannya. Terakhir, merupakan argumen wandb yang digunakan untuk konfigurasi sinkronisasi pada hasil evaluasi ke \textit{cloud}.

\subsection{Praproses \textit{Dataset}}
\label{sec:praproses}

Praproses \textit{dataset} diperlukan agar model dapat memproses \textit{dataset} dengan konteks yang sesuai. Konteks yang dimaksud adalah, model dapat memahami \textit{metadata} dari \textit{dataset}, sehingga mampu mengetahui setiap bagian data yang merupakan bagian dari teks atau label. Praproses \textit{dataset} ini berbeda untuk setiap tugas evaluasi karena karakteristik dari tugas evaluasinya yang berbeda juga.

\subsubsection{Praproses Data NER}

Tugas evaluasi NER merupakan jenis tugas \textit{classification} pada level token yang berarti setiap token akan diklasifikasikan dengan label tertentu. Pada kakas IndoLEM, untuk tugas evaluasi NER terdapat dua jenis \textit{dataset}, yaitu NERUI dan NERUGM. Kedua \textit{dataset} tersebut mempunyai label yang berbeda berikut merupakan label pada setiap \textit{dataset}-nya dapat dilihat pada tabel \ref{table:label-ner}.

\begin{table}[h]
    \vspace{0.25cm}
    \caption{Tabel label \textit{dataset} NER}
    \label{table:label-ner}
    \begin{center}
        \begin{tabular}{|l|l|}
            \hline \rowcolor{black!10}
            \multicolumn{1}{|c|}{\textbf{NERUI}} & \multicolumn{1}{|c|}{\textbf{NERUI}} \\ \hline
            B-LOCATION & B-LOCATION \\ \hline
            B-ORGANIZATION & B-ORGANIZATION \\ \hline
            B-PERSON & B-PERSON \\ \hline
            - & B-QUANTITY \\ \hline
            - & B-TIME \\ \hline
            I-LOCATION & I-LOCATION \\ \hline
            I-ORGANIZATION & I-ORGANIZATION \\ \hline
            I-PERSON & I-PERSON \\ \hline
            - & I-QUANTITY \\ \hline
            - & I-TIME \\ \hline
            O & O \\ \hline
        \end{tabular}
    \end{center}
\end{table}

Label tersebut diperlukan untuk melakukan \textit{mapping} antara id dengan labelnya yang akan digunakan pada konfigurasi model. \textit{Padding} juga dilakukan berdasarkan panjang sekuens maksimum yang didefinisikan melaluai argumen \textit{data training}. Selanjutnya, tokenisasi akan dilakukan terhadap \textit{dataset} tersebut. Implementasi dari fungsi tokenisasi tersebut dapat dilihat pada tabel \ref{table:tokenisasi-ner}.

\begin{table}[h]
    \caption{Tabel fungsi tokenisasi NER}
    \label{table:tokenisasi-ner}
    \begin{lstlisting}[language=python]
    def tokenize_and_align_labels(examples):
        tokenized_inputs = tokenizer(
            examples[text_column_name],
            padding=padding,
            truncation=True,
            max_length=data_args.max_seq_length,
            is_split_into_words=True,
        )
        labels = []

        for i, label in enumerate(examples[label_column_name]):
            word_ids = tokenized_inputs.word_ids(batch_index=i)
            previous_word_idx = None
            label_ids = []
            for word_idx in word_ids:
                if word_idx is None:
                    label_ids.append(-100)
                elif word_idx != previous_word_idx:
                    label_ids.append(label_to_id[label[word_idx]])
                else:
                    if data_args.label_all_tokens:
                        label_ids.append(b_to_i_label[label_to_id[label[word_idx]]])
                    else:
                        label_ids.append(-100)
                previous_word_idx = word_idx

            labels.append(label_ids)
        tokenized_inputs["labels"] = labels
        return tokenized_inputs
    \end{lstlisting}
\end{table}

Tokenisasi dilakukan dengan menggunakan modul Tokenizer pada pustaka Transformers. Fungsi tokenisasi pada tabel \ref{table:tokenisasi-ner} mengabaikan token spesial [CLS] dan [SEP] yang muncul dari hasil tokenisasi model berbasi BERT. Token spesial tersebut diubah menjadi -100 agar bisa diabaikan dari perhitungan metriks evaluasinya. Sehingga, hanya token yang berkorespondensi terhadap sebuah label saja yang dapat dievaluasi.

\subsubsection{Praproses Data \textit{Sentiment Analysis}}

Tugas evaluasi \textit{sentiment analysis} merupakan jenis tugas \textit{classification} sama seperti NER, hanya perbedaannya pada level klasifikasi yang digunakan. Jika pada NER klasifikasi dilakukan pada level token, pada \textit{sentiment analysis} klasifikasi dilakukan pada level teks yang merupakan gabungan dari beberapa token. \textit{Dataset} dari \textit{sentiment analysis} berupa \textit{file} CSV yang berisi dari dua kolom yaitu \textit{sentence} dan \textit{labels}. Kolom \textit{sentence} merupakan teks yang mengandung sentiment tertentu. Sementara kolom \textit{labels} menandakan sentimen dari teks tersebut, dengan nilai 0 berupa sentimen negatif dan  nilai 1 berupa sentiment positif.

\begin{table}[h]
    \caption{Tabel fungsi tokenisasi \textit{sentiment analysis}}
    \label{table:tokenisasi-sentiment}
    \begin{lstlisting}[language=python]
        def preprocess_function(examples):
            # Tokenize the texts
            args = (
                (examples[sentence1_key],)
                if sentence2_key is None
                else (examples[sentence1_key], examples[sentence2_key])
            )
            result = tokenizer(
                *args, padding=padding, max_length=max_seq_length, truncation=True
            )

            # Map labels to IDs
            if label_to_id is not None and "labels" in examples:
                result["labels"] = [
                    (label_to_id[l] if l != -1 else -1) for l in examples["labels"]
                ]
            return result
    \end{lstlisting}
\end{table}

Pada tugas evaluasi \textit{sentiment analysis} juga diperlukan \textit{mapping} antara label dengan idnya, sehingga model mampu mengetahui label yang dilatih. Fungsi tokenisasi \textit{sentiment analysis} yang dapat dilihat pada tabel \ref{table:tokenisasi-sentiment} menggunakan modul Tokenizer dari pustaka Transformers. Fungsi tersebut terdapat \textit{sentence1\_key} dan juga \textit{sentence2\_key}, yang berguna apabila terdapat dua kolom teks pada \textit{dataset}, tetapi pada IndoLEM \textit{dataset sentiment analysis} hanya mempunyai satu kolom teks. Sehingga, \textit{sentence2\_key} tidak digunakan. 

\subsubsection{Praproses Data \textit{Summarization}}

Tugas evaluasi \textit{summarization} merupakan jenis tugas \textit{generation} yang berbeda dengan jenis tugas \textit{classification}. Tugas \textit{generation} memerlukan generasi teks untuk dapat dilakukan evaluasi, pada konteks ini yaitu membuat ringkasan dari teks. \textit{Dataset} yang digunakan pada IndoLEM untuk tugas evaluasi \textit{summarization} adalah \textit{dataset} IndoSUM. \textit{Dataset} ini berisi \textit{file} dengan format JSONL dengan masing-masing objek dapat dilihat pada tabel \ref{table:data-summarization}.

\begin{table}[h]
    \vspace{0.25cm}
    \caption{Tabel data \textit{summarization}}
    \label{table:data-summarization}
    \begin{center}
        \begin{tabular}{|l|l|}
            \hline \rowcolor{black!10}
            \multicolumn{1}{|c|}{\textbf{Objek}} & \multicolumn{1}{|c|}{\textbf{Deskripsi}} \\ \hline
            id & nilai unik untuk setiap artikel \\ \hline
            paragraphs & \textit{list} dari paragraf yang berasal dari artikel \\ \hline
            summary & ringkasan dari artikel, disimpan sebagai \textit{list} dari kalimat \\ \hline 
            gold\_labels & label ekstraktif dari setiap kalimat pada artikel \\ \hline
            category & kategori dari artikel \\ \hline
            source & sumber artikel \\ \hline
            source\_url & URL dari artikel \\ \hline
        \end{tabular}
    \end{center}
\end{table}

Dari data yang dapat dilihat pada tabel \ref{table:data-summarization}, terdapat 7 objek  pada \textit{dataset}. Namun, pada tugas evaluasi ini hanya akan digunakan dua objek yaitu \textit{paragraphs} dan \textit{summary} karena tugas \textit{summarization} membutuhkan teks asli sebelum diringkas dan hasil ringkasannya. Objek yang lain selain dua yang digunakan bisa diabaikan saja. \textit{Dataset} ini perlu dipraproses terlebih dahulu menggunakan fungsi yang dapat dilihat pada tabel \ref{table:tokensisasi-summ}.

\begin{table}[h]
    \caption{Tabel fungsi tokenisasi \textit{summarization}}
    \label{table:tokensisasi-summ}
    \begin{lstlisting}[language=python]
    def preprocess_function(examples):
        if is_indosum:
            examples[text_column] = paragraph_to_text(examples[text_column])
            examples[summary_column] = summary_to_text(examples[summary_column])

        inputs = [prefix + inp for inp in inputs]
        model_inputs = tokenizer(
            inputs,
            max_length=data_args.max_source_length,
            padding=padding,
            truncation=True,
        )

        labels = tokenizer(
            text_target=targets,
            max_length=max_target_length,
            padding=padding,
            truncation=True,
        )

        if padding == "max_length" and data_args.ignore_pad_token_for_loss:
            labels["input_ids"] = [
                [(l if l != tokenizer.pad_token_id else -100) for l in label]
                for label in labels["input_ids"]
            ]

        model_inputs["labels"] = labels["input_ids"]
        return model_inputs
    \end{lstlisting}
\end{table}

Berdasarkan tabel \ref{table:tokensisasi-summ}, tokenisasi dilakukan dengan mengolah objek \textit{paragraphs} menjadi \textit{list} dari teks. Setiap paragraf terdiri dari beberapa kalimat, dan setiap kalimat terdiri dari beberapa teks. Untuk \textit{text\_column} yang merupakan objek \textit{paragraphs} akan diolah menjadi \textit{list} dari teks. Hal yang sama berlaku juga untuk \textit{summary\_column}, hanya saja objek ini merupakan \textit{summary} yang berisi kalimat. Selain itu, tugas evaluasi \textit{summarization} ini terdapat \textit{prefix} yang berguna pada beberapa model tertentu. Tokenisasi juga menggunakan modul \textit{tokenizer} dari pustaka Transformers.

\subsection{Pemuatan Konfigurasi Model}

Untuk dapat melakukan pelatihan dan evaluasi terhadap model, terlebih dahulu model perlu dimuat. Pemuatan model ini menggunakan konfigurasi yang dimuat dari penguraian argumen pada subbab \ref{sec:parse-arg}. Argumen model akan dimuat ke dalam model sehingga model bisa dipakai. Model akan dimuat dengan menggunakan modul yang relevan dari pustaka Transformers.

\begin{table}[h]
    \caption{Tabel kode pemuatan model \textit{summarization}}
    \label{table:kode-model-summ}
    \begin{lstlisting}[language=python]
    model = AutoModelForSeq2SeqLM.from_pretrained(
        model_args.model_name_or_path,
        from_tf=bool(".ckpt" in model_args.model_name_or_path),
        config=config,
        cache_dir=model_args.cache_dir,
        revision=model_args.model_revision,
        token=True if model_args.token else None,
    )
    \end{lstlisting}
\end{table}

Pada tabel \ref{table:kode-model-summ} dimuat model AutoModelForSeq2SeqLM yang merupakan model \textit{sequence to sequence} yang merupakan model \textit{encoder decoder}. Modul AutoModel ini akan menghasilkan model berdasarkan dari konfigurasi \textit{model\_args} yang dimasukkan pada modul tersebut. Untuk tugas evaluasi \textit{summarization} yang menggunakan model IndoT5 akan menghasilkan model tersebut juga. Berbeda untuk tugas \textit{classification}, cukup dengan menyatakan AutoModel saja sudah cukup untuk menghasilkan model yang sesuai. Pada konteks ini, model yang digunakan untuk NER dan \textit{sentiment analysis} adalah IndoBERT yang berbasis BERT.

\subsection{Perhitungan Metriks Evaluasi}
\label{sec:metriks-evaluasi}

Metriks evaluasi yang digunakan pada setiap tugas evaluasi tidak sama semuanya. Metriks yang digunakan untuk tugas \textit{classification} adalah \textit{accuracy} dan F1 \textit{score}. Sedangkan, pada tugas \textit{generation} adalah ROUGE \textit{score}. Perhitungan metriks evaluasi ini dilakukan saat proses evaluasi dengan menggunakan fungsi \textit{compute\_metrics}.

\begin{table}[h]
    \caption{Tabel fungsi \textit{compute\_metrics}}
    \label{table:compute-metrics}
    \begin{lstlisting}[language=python]
    def compute_metrics(p):
        logits, labels = p
        predictions = np.argmax(logits, axis=1)

        return {
            "accuracy": accuracy_score(labels, predictions),
            "precision": precision_score(labels, predictions, average="macro"),
            "recall": recall_score(labels, predictions, average="macro"),
            "f1": f1_score(labels, predictions, average="macro"),
        }
    \end{lstlisting}
\end{table}

Pada tabel \ref{table:compute-metrics}, fungsi perhitungan metriks evaluasinya merupakan fungsi yang dipakai untuk tugas \textit{classification}. Fungsi tersebut memanfaatkan modul seqeval untuk melakukan perhitungan setiap metriksnya. Terdapat perbeedaan pada fungsi yang dipakai pada tugas \textit{classifcation}, NER menggunakan \textit{entity\_level\_metrics} yang berarti perhitungan evaluasinya berdasarkan pada entitasnya yang berbeda dengan \textit{sentiment analysis}. Untuk tugas \textit{generation} yaitu \textit{summarization} menggunakan ROUGE \textit{score}.

\subsection{Pemuatan Trainer untuk Pelatihan dan Evaluasi Model}

Untuk pelatihan dan evaluasi model dibutuhkan standardisasi agar kode yang digunakan pada setiap tugas tidak berbeda. Modul Trainer dari pustaka Transformers digunakan untuk pelatihan dan evaluasi model. Setiap komponen yang sudah disebutkan sebelumnya akan dimuat pada modul Trainer. Setiap tugas evaluasi akan menggunakan implementasi yang sama yang bisa dilihat pada tabel \ref{table:kode-trainer}.

\begin{table}[h]
    \caption{Tabel kode implementasi Trainer}
    \label{table:kode-trainer}
    \begin{lstlisting}[language=python]
    trainer = Trainer(
        model=model,
        args=training_args,
        train_dataset=train_dataset if training_args.do_train else None,
        eval_dataset=eval_dataset if training_args.do_eval else None,
        tokenizer=tokenizer,
        data_collator=data_collator,
        compute_metrics=compute_metrics,
    )

    trainer.train(resume_from_checkpoint=checkpoint)
    trainer.evaluate()
    \end{lstlisting}
\end{table}

Kode implementasi Trainer pada tabel \ref{table:kode-trainer} memuat modul Trainer dan memanggil metode \textit{train} dan \textit{evaluate} untuk melakukan pelatihan dan evaluasi. Modul Trainer membutuhkan beberapa argumen yang dimuat. Argumen tersebut dimuat dari komponen-komponen yang sudah disebutkan pada subbab sebelumnya. Argumen untuk pelatihan, dan model didapat dari hasil penguraian argumen pada \ref{sec:parse-arg}. Untuk \textit{tokenizer} dan \textit{dataset} pelatihan dan evaluasi dimuat dari hasil praproses data pada \ref{sec:praproses}. Lalu, untuk \textit{compute\_metrics} fungsinya dimuat dari metriks evaluasi pada \ref{sec:metriks-evaluasi}. 

\section{Pengembangan Metode PEFT pada Kakas IndoLEM}

Pengembangan metode PEFT pada kakas IndoLEM menggunakan pustaka Adapters. Pustaka tersebut tidak hanya mengimplementasikan setiap metode saja, tetapi terdapat fungsionalitas untuk menggabungkan beberapa metode PEFT. Pada subbab sebelumnya yaitu subbab \ref{sec:pengembangan-kakas}, kakas IndoLEM dikembangkan ulang agar mampu untuk mengimplementasikan metode PEFT. Implementasi tersebut  mengubah beberapa komponen, yaitu komponen model dan trainer. Metode PEFT  melakukan \textit{freeze} pada parameter model sehingga pada proses pelatihan tidak  mengubah parameter model melainkan parameter dari metode PEFT tersebut.

\begin{table}[h]
    \caption{Tabel kode pemuatan model Adapters}
    \label{table:kode-model-adapters}
    \begin{lstlisting}[language=python]
    model = AutoAdapterModel.from_pretrained(
        model_args.model_name_or_path,
        from_tf=bool(".ckpt" in model_args.model_name_or_path),
        config=config,
        cache_dir=model_args.cache_dir,
        revision=model_args.model_revision,
        token=True if model_args.token else None,
        ignore_mismatched_sizes=model_args.ignore_mismatched_sizes,
    )
    \end{lstlisting}
\end{table}

Pada kode pemuatan model Adapters yang bisa dilihat pada tabel \ref{table:kode-model-adapters}, model dimuat dengan modul AutoAdapterModel dari pustaka Adapters. Hal ini dilakukan untuk \textit{support} metode PEFT yang lebih baik berdasarkan dari dokumentasi dari pustakanya. Berbeda dari tabel \ref{table:kode-model-summ} yang menggunakan AutoModel dari pustaka Transformers, AutoAdapterModel memberikan fungsionalitas yang lebih baik untuk metode PEFT, salah satunya adalah untuk membandingkan parameter yang  dilatih dibanding dengan parameter asli dari modelnya.

\begin{table}[h]
    \caption{Tabel kode implementasi AdapterTrainer}
    \label{table:kode-adaptertrainer}
    \begin{lstlisting}[language=python]
    setup_adapter_training(model, adapter_args)
    trainer = AdapterTrainer(
        model=model,
        args=training_args,
        train_dataset=train_dataset if training_args.do_train else None,
        eval_dataset=eval_dataset if training_args.do_eval else None,
        tokenizer=tokenizer,
        data_collator=data_collator,
        compute_metrics=compute_metrics,
    )

    trainer.train(resume_from_checkpoint=checkpoint)
    trainer.evaluate()
    \end{lstlisting}
\end{table}

Selanjutnya, untuk proses pelatihan perlu disiapkan untuk menggunakan metode PEFT. Implementasinya menggunakan modul AdapterTrainer dari pustaka Adapters juga. Model perlu disiapkan untuk pelatihan dengan metode PEFT dengan menggunakan fungsi \textit{setup\_adapter\_training}. Fungsi tersebut  melakukan \textit{freeze} pada parameter model dan hanya  mengubah parameter dari metode PEFT yang digunakan. Argumen untuk \textit{adapter} dimuat dari hasil pengurain argumen pada \ref{sec:parse-arg}, argumen tersebut berisi jenis metode PEFT serta \textit{hyperparameter} yang digunakan pada proses pelatihan. Proses pelatihan metode PEFT berbeda dengan pelatihan dengan metode \textit{fine-tuning} tergantung dari karakteristik dari metode tersebut. Pustaka Adapters mengimplementasikannya dengan menambahkan semacam \textit{adapter} pada model, sehingga \textit{adapter} tersebut yang dilatih bukan modelnya. Parameter dari \textit{adapter} tersebut relatif lebih kecil daripada parameter model.

\section{Pelatihan Model}

Pelatihan model dilakukan pada lingkungan eksperimen yang telah disebutkan pada subbab \ref{sec:lingkungan-eksperimen}. Pelatihan dilakukan dengan \textit{bash script} yang  dijalankan pada \textit{shell} untuk memudahkan reka ulang dari eksperimen tersebut. Untuk setiap tugas evaluasi terdapat \textit{bash script} yang  menjalankan semua eksperimen, yaitu \textit{fine-tuning}, \methodPEFT.

\begin{figure}[h]
    \centering
    \caption{Struktur \textit{file} eksperimen}
    \label{fig:file-eksperimen}
    \begin{forest}
        for tree={
            font=\ttfamily,
            grow'=0,
            child anchor=west,
            parent anchor=south,
            anchor=west,
            calign=first,
            edge path={
                \noexpand\path [draw, \forestoption{edge}]
                (!u.south west) +(7.5pt,0) |- node[fill,inner sep=1.25pt] {} (.child anchor)\forestoption{edge label};
            },
            before typesetting nodes={
                if n=1
                    {insert before={[,phantom]}}
                    {}
            },
            fit=band,
            before computing xy={l=15pt},
        }
    [run\_sentiment.sh
        [script
            [run\_sentiment\_base.sh]
            [run\_sentiment\_lora.sh]
            [run\_sentiment\_pt.sh]
            [run\_sentiment\_seq\_bn.sh]
            [run\_sentiment\_unipelt.sh]
        ]
    ]
    \end{forest}
\end{figure}

Pada gambar \ref{fig:file-eksperimen}, dengan melakukan pemanggilan pada {\ttfamily run\_sentiment.sh}  menjalankan semua eksperimen yang ada pada folder {\ttfamily script}. Setiap \textit{script}  menjalankan eksperimen sesuai dari konfigurasi yang telah ditentukan pada \textit{script} tersebut. Konfigurasi yang ditentukan tersebut  diterima sebagai argumen pada kakas IndoLEM. Untuk setiap \textit{hyperparameter} yang digunakan untuk pelatihan dapat dilihat pada tabel \ref{table:hyperparameter-train}.

\begin{table}[h]
    \centering
    \caption{Tabel \textit{hyperparameter} pelatihan}
    \label{table:hyperparameter-train}
    \begin{tabular}{|c|l|c|}
        \hline \rowcolor{black!10}
        \textbf{Tugas} & \multicolumn{1}{|c|}{\textbf{Argumen}} & \textbf{Nilai} \\ \hline
        \multirow{9}{*}{\textit{NER}} & model & "indolem/indobert-base-uncased" \\ \cline{2-3}
                                      & train batch size & 16 \\ \cline{2-3}
                                      & eval batch size & 64 \\ \cline{2-3}
                                      & epochs & 100 \\ \cline{2-3}
                                      & learning rate & 5e-5 \\ \cline{2-3}
                                      & max sequence length & 128 \\ \cline{2-3}
                                      & text column names & "tokens" \\ \cline{2-3}
                                      & label column names & "ner\_tags" \\ \cline{2-3}
                                      & return entity level metrics & true \\ \cline{2-3}
                                      & seed & 42 \\ \hline
        \multirow{7}{*}{\textit{sentiment analysis}} & model & "indolem/indobert-base-uncased" \\ \cline{2-3}
                                                     & batch size & 30 \\ \cline{2-3}
                                                     & epochs & 20 \\ \cline{2-3}
                                                     & learning rate & 5e-5 \\ \cline{2-3}
                                                     & max sequence length & 200 \\ \cline{2-3}
                                                     & seed & 42 \\ \cline{2-3}
                                                     & label names & "labels" \\ \hline
        \multirow{14}{*}{\textit{summarization}} & model & "LazarusNLP/IndoNanoT5-base" \\ \cline{2-3}
                                                & train batch size & 4 \\ \cline{2-3}
                                                & eval batch size & 8 \\ \cline{2-3}
                                                & epochs & 5 \\ \cline{2-3}
                                                & learning rate & 1e-5 \\ \cline{2-3}
                                                & max source length & 512 \\ \cline{2-3}
                                                & max target length & 512 \\ \cline{2-3}
                                                & num beams & 5 \\ \cline{2-3}
                                                & weight decay & 0.01 \\ \cline{2-3}
                                                & patience & 5 \\ \cline{2-3}
                                                & seed & 42 \\ \cline{2-3}
                                                & text column & paragraphs \\ \cline{2-3}
                                                & summary column & summary \\ \cline{2-3}
                                                & source prefix & "summarize: " \\ \cline{2-3}
                                                & pad to max length & true \\ \cline{2-3}
                                                & predict with generate & true \\ \cline{2-3}
                                                & bf16 & true \\ \hline
    \end{tabular}
\end{table}

Untuk pelatihan model NER terdapat dua \textit{dataset} yang  digunakan yaitu NERUGM dan NERUI. Sedangkan, untuk tugas \textit{sentiment analysis} dan \textit{summarization} menggunakan satu dataset. Selain itu, untuk tugas NER terdapat argumen "return entity level metrics" dengan nilai "true" yang berarti penilaian metriks evaluasinya  dinilai pada level entitas. Ini mengikuti eksperimen yang sebelumnya sudah dilakukan pada IndoLEM.

Terdapat perbedaan pada \textit{hyperparameter} yang digunakan untuk tugas \textit{summarization} karena argumen yang digunakan mengikuti eksperimen pada IndoT5 bukan pada IndoLEM. Terdapat juga argumen "predict with generate" dengan nilai "true" karena tugas \textit{summarization} perlu menghasilkan hasil ringkasan untuk dapat dievaluasi. Selain itu, nilai argumen "bf16" bernilai "true" karena presisi "bf16" dianggap bisa mempercepat proses pelatihan untuk model T5 yang digunakan. Pada tugas ini perlu menambahkan argumen "source prefix" yang bernilai "summarize: " khusus unutk penggunaan model T5 pada tugas \textit{summarization}.

\begin{table}[h]
    \centering
    \caption{Tabel \textit{hyperparameter} metode PEFT}
    \label{table:hyperparameter-PEFT}
    \begin{tabular}{|l|l|c|}
        \hline \rowcolor{black!10}
        \multicolumn{1}{|c|}{\textbf{Metode}} & \multicolumn{1}{|c|}{\textbf{Argumen}} & \textbf{Nilai} \\ \hline
        LoRA & ranks & [8, 16] \\ \hline
        Prefix Tuning & prefix length & [10, 20, 30] \\ \hline
        Bottleneck Adapter & \multicolumn{1}{|c|}{-} & - \\ \hline
        UniPELT & \multicolumn{1}{|c|}{-}  & - \\ \hline
    \end{tabular}
\end{table}

Pada pelatihan dengan metode PEFT  menggunakan \textit{hyperparameter} pelatihan seperti pada tabel \ref{table:hyperparameter-train}. Untuk setiap metode PEFT terdapat \textit{hyperparameter} yang sesuai dengan metode PEFT tersebut. Perlu dilakukan \textit{hyperparameter tuning} untuk \textit{hyperparameter} metode PEFT. Seperti yang bisa dilihat pada tabel \ref{table:hyperparameter-PEFT}, untuk metode LoRA dilakukan eksperimen dengan argumen "ranks" dengan nilai 8 dan 16. Lalu, untuk metode Prefix Tuning dilakukan eksperimen dengan argumen "prefix tuning" dengan nilai 10, 20, dan 30. Selain itu, yaitu metode Bottleneck Adapter dan UniPELT tidak menggunakan argumen tambahan, hal ini dilakukan karena metode tersebut memang spesifik diimplementasikan pada pustaka Adapters sesuai dengan penelitian terkaitnya.

\section{Evaluasi Model}

