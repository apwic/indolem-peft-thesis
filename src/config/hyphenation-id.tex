%--------------------------------------------------------------------%
%
% Hyphenation untuk Bahasa Indonesia
%
% @author Petra Barus
%
%--------------------------------------------------------------------%
%
% Secara otomatis LaTeX dapat langsung memenggal kata dalam dokumen,
% tapi sering kali terdapat kesalahan dalam pemenggalan kata. Untuk
% memperbaiki kesalahan pemenggalan kata tertentu, cara pemenggalan
% kata tersebut dapat ditambahkan pada dokumen ini. Pemenggalan
% dilakukan dengan menambahkan karakter '-' pada suku kata yang
% perlu dipisahkan.
%
% Contoh pemenggalan kata 'analisa' dilakukan dengan 'a-na-li-sa'
%
%--------------------------------------------------------------------%

\hyphenation {
	% A
	%
	a-na-li-sa
	a-pli-ka-si

	% B
	%
	be-be-ra-pa
	ber-ge-rak
	bah-kan

	% C
	%
	ca-ri

	% D
	%
	da-e-rah
	di-nya-ta-kan
	de-fi-ni-si
	di-gu-na-kan
	di-da-pat-kan

	% E
	%
	e-ner-gi
	eks-klu-sif

	% F
	%
	fa-si-li-tas

	% G
	%
	ga-bung-an

	% H
	%
	ha-lang-an

	% I
	% 
	i-nduk

	% J
	%
	ka-me-ra
	kua-li-tas

	% K
	%
	ke-ung-gul-an

	% L
	%
	ling-ku-ngan

	% M
	%
	me-ngan-dung
	me-man-fa-at-kan
	mem-ban-ding-kan
	men-ja-lan-kan
	meng-gu-na-kan

	% N
	%

	% O
	%

	% P
	%
	pe-la-ti-han

	% Q
	%

	% R
	%

	% S
	%

	% T
	% 

	% U
	%

	% V
	%

	% W
	%

	% X
	%

	% Y
	% 

	% Z
	%
}
